% Options for packages loaded elsewhere
\PassOptionsToPackage{unicode}{hyperref}
\PassOptionsToPackage{hyphens}{url}
\PassOptionsToPackage{dvipsnames,svgnames,x11names}{xcolor}
%
\documentclass[
  super,
  preprint,
  3p]{elsarticle}

\usepackage{amsmath,amssymb}
\usepackage{setspace}
\usepackage{iftex}
\ifPDFTeX
  \usepackage[T1]{fontenc}
  \usepackage[utf8]{inputenc}
  \usepackage{textcomp} % provide euro and other symbols
\else % if luatex or xetex
  \usepackage{unicode-math}
  \defaultfontfeatures{Scale=MatchLowercase}
  \defaultfontfeatures[\rmfamily]{Ligatures=TeX,Scale=1}
\fi
\usepackage{lmodern}
\ifPDFTeX\else  
    % xetex/luatex font selection
  \setmainfont[]{Comic Sans MS}
\fi
% Use upquote if available, for straight quotes in verbatim environments
\IfFileExists{upquote.sty}{\usepackage{upquote}}{}
\IfFileExists{microtype.sty}{% use microtype if available
  \usepackage[]{microtype}
  \UseMicrotypeSet[protrusion]{basicmath} % disable protrusion for tt fonts
}{}
\makeatletter
\@ifundefined{KOMAClassName}{% if non-KOMA class
  \IfFileExists{parskip.sty}{%
    \usepackage{parskip}
  }{% else
    \setlength{\parindent}{0pt}
    \setlength{\parskip}{6pt plus 2pt minus 1pt}}
}{% if KOMA class
  \KOMAoptions{parskip=half}}
\makeatother
\usepackage{xcolor}
\usepackage{soul}
\setlength{\emergencystretch}{3em} % prevent overfull lines
\setcounter{secnumdepth}{5}
% Make \paragraph and \subparagraph free-standing
\ifx\paragraph\undefined\else
  \let\oldparagraph\paragraph
  \renewcommand{\paragraph}[1]{\oldparagraph{#1}\mbox{}}
\fi
\ifx\subparagraph\undefined\else
  \let\oldsubparagraph\subparagraph
  \renewcommand{\subparagraph}[1]{\oldsubparagraph{#1}\mbox{}}
\fi


\providecommand{\tightlist}{%
  \setlength{\itemsep}{0pt}\setlength{\parskip}{0pt}}\usepackage{longtable,booktabs,array}
\usepackage{calc} % for calculating minipage widths
% Correct order of tables after \paragraph or \subparagraph
\usepackage{etoolbox}
\makeatletter
\patchcmd\longtable{\par}{\if@noskipsec\mbox{}\fi\par}{}{}
\makeatother
% Allow footnotes in longtable head/foot
\IfFileExists{footnotehyper.sty}{\usepackage{footnotehyper}}{\usepackage{footnote}}
\makesavenoteenv{longtable}
\usepackage{graphicx}
\makeatletter
\def\maxwidth{\ifdim\Gin@nat@width>\linewidth\linewidth\else\Gin@nat@width\fi}
\def\maxheight{\ifdim\Gin@nat@height>\textheight\textheight\else\Gin@nat@height\fi}
\makeatother
% Scale images if necessary, so that they will not overflow the page
% margins by default, and it is still possible to overwrite the defaults
% using explicit options in \includegraphics[width, height, ...]{}
\setkeys{Gin}{width=\maxwidth,height=\maxheight,keepaspectratio}
% Set default figure placement to htbp
\makeatletter
\def\fps@figure{htbp}
\makeatother

\usepackage{lineno}\linenumbers
\makeatletter
\makeatother
\makeatletter
\makeatother
\makeatletter
\@ifpackageloaded{caption}{}{\usepackage{caption}}
\AtBeginDocument{%
\ifdefined\contentsname
  \renewcommand*\contentsname{Table of contents}
\else
  \newcommand\contentsname{Table of contents}
\fi
\ifdefined\listfigurename
  \renewcommand*\listfigurename{List of Figures}
\else
  \newcommand\listfigurename{List of Figures}
\fi
\ifdefined\listtablename
  \renewcommand*\listtablename{List of Tables}
\else
  \newcommand\listtablename{List of Tables}
\fi
\ifdefined\figurename
  \renewcommand*\figurename{Figure}
\else
  \newcommand\figurename{Figure}
\fi
\ifdefined\tablename
  \renewcommand*\tablename{Table}
\else
  \newcommand\tablename{Table}
\fi
}
\@ifpackageloaded{float}{}{\usepackage{float}}
\floatstyle{ruled}
\@ifundefined{c@chapter}{\newfloat{codelisting}{h}{lop}}{\newfloat{codelisting}{h}{lop}[chapter]}
\floatname{codelisting}{Listing}
\newcommand*\listoflistings{\listof{codelisting}{List of Listings}}
\makeatother
\makeatletter
\@ifpackageloaded{caption}{}{\usepackage{caption}}
\@ifpackageloaded{subcaption}{}{\usepackage{subcaption}}
\makeatother
\makeatletter
\@ifpackageloaded{tcolorbox}{}{\usepackage[skins,breakable]{tcolorbox}}
\makeatother
\makeatletter
\@ifundefined{shadecolor}{\definecolor{shadecolor}{rgb}{.97, .97, .97}}
\makeatother
\makeatletter
\makeatother
\makeatletter
\makeatother
\journal{Elsevier}
\ifLuaTeX
  \usepackage{selnolig}  % disable illegal ligatures
\fi
\usepackage[]{natbib}
\bibliographystyle{elsarticle-num}
\IfFileExists{bookmark.sty}{\usepackage{bookmark}}{\usepackage{hyperref}}
\IfFileExists{xurl.sty}{\usepackage{xurl}}{} % add URL line breaks if available
\urlstyle{same} % disable monospaced font for URLs
\hypersetup{
  pdftitle={Development of Chemical Categories for Per- and Polyfluoroalkyl Substances (PFAS) and the Proof-of-Concept Approach to the Identification of Potential Candidates for Tiered Toxicological Testing and Human Health Assessment},
  pdfauthor={Grace Patlewicz; Richard Judson; Antony Williams; Tristan Butler; Stan Barone Jr.; Kelly Carstens; John Cowden; Jeffrey L. Dawson; Sigmund Degitz; Kellie Fay; Tala R. Henry; Anna Lowit; Stephanie Padilla; Katie Paul Friedman; Martin B. Phillips; David Turk; John Wambaugh; Barbara Wetmore; Russell S. Thomas},
  pdfkeywords={Per- and polyfluoroalkyl substances (PFAS), chemical
categories, read-across, New Approach Methods (NAMs), data
collection, Toxic Substances Control Act (TSCA)},
  colorlinks=true,
  linkcolor={blue},
  filecolor={Maroon},
  citecolor={Blue},
  urlcolor={Blue},
  pdfcreator={LaTeX via pandoc}}

\setlength{\parindent}{6pt}
\begin{document}

\begin{frontmatter}
\title{Development of Chemical Categories for Per- and Polyfluoroalkyl
Substances (PFAS) and the Proof-of-Concept Approach to the
Identification of Potential Candidates for Tiered Toxicological Testing
and Human Health Assessment}
\author[1]{Grace Patlewicz%
\corref{cor1}%
}
 \ead{patlewicz.grace@epa.gov} 
\author[1]{Richard Judson%
%
}

\author[1]{Antony Williams%
%
}

\author[2]{Tristan Butler%
%
}

\author[2]{Stan Barone Jr.%
%
}

\author[1]{Kelly Carstens%
%
}

\author[1]{John Cowden%
%
}

\author[2]{Jeffrey L. Dawson%
%
}

\author[1]{Sigmund Degitz%
%
}

\author[2]{Kellie Fay%
%
}

\author[2]{Tala R. Henry%
%
\fnref{fn1}}

\author[2]{Anna Lowit%
%
}

\author[1]{Stephanie Padilla%
%
}

\author[1]{Katie Paul Friedman%
%
}

\author[2]{Martin B. Phillips%
%
}

\author[2]{David Turk%
%
}

\author[1]{John Wambaugh%
%
}

\author[1]{Barbara Wetmore%
%
}

\author[1]{Russell S. Thomas%
%
}


\affiliation[1]{organization={Center for Computational Toxicology and
Exposure (CCTE), US Environmental Protection
Agency},city={Durham},country={USA},countrysep={,},postcodesep={}}
\affiliation[2]{organization={Office of Chemical Safety and Pollution
Prevention (OSCPP), US Environmental Protection
Agency},city={DC},country={USA},countrysep={,},postcodesep={}}

\cortext[cor1]{Corresponding author}










\fntext[fn1]{Retired}








        
\begin{abstract}
Per- and Polyfluoroalkyl substances (PFAS) are a class of manufactured
chemicals that are in widespread use and many present concerns for
persistence, bioaccumulation and toxicity. Whilst a handful of PFAS have
been characterized for their hazard profiles, the vast majority have not
been extensively studied. Herein, a chemical category approach was
developed and applied to PFAS that could be readily characterized by a
chemical structure. The PFAS definition as described in the Toxic
Substances Control Act (TSCA) section 8(a)(7) rule was applied to the
Distributed Structure-Searchable Toxicity (DSSTox) database to retrieve
an initial list of 13,054 PFAS. Plausible degradation products from the
563 PFAS on the non-confidential TSCA Inventory were simulated using the
Catalogic expert system, and the unique predicted PFAS degradants (2484)
that conformed to the same PFAS definition were added to the list
resulting in a set of 15,538 PFAS. Each PFAS was then assigned into a
primary category using Organisation for Economic Co-operation and
Development (OECD) structure-based classifications. The primary
categories were subdivided into secondary categories based on a chain
length threshold (\textgreater=7 vs \textless7). Secondary categories
were subcategorized using chemical fingerprints to achieve a balance
between total number of structural categories vs.~level of structural
similarity within a category based on the Jaccard index. A set of 128
terminal structural categories were derived from which a subset of
representative candidates could be proposed for potential data
collection, considering the sparsity of relevant toxicity data within
each category, presence on environmental monitoring lists, and the
ability to identify plausible manufacturers/importers. Refinements to
the approach taking into consideration ways in which the categories
could be updated by mechanistic data and physicochemical property
information are also described. This categorization approach may be used
to form the basis of identifying candidates for data collection with
related applications in QSAR development, read-across and hazard
assessment.
\end{abstract}





\begin{keyword}
    Per- and polyfluoroalkyl substances (PFAS) \sep chemical
categories \sep read-across \sep New Approach Methods (NAMs) \sep data
collection \sep 
    Toxic Substances Control Act (TSCA)
\end{keyword}
\end{frontmatter}
    \ifdefined\Shaded\renewenvironment{Shaded}{\begin{tcolorbox}[sharp corners, enhanced, boxrule=0pt, interior hidden, breakable, frame hidden, borderline west={3pt}{0pt}{shadecolor}]}{\end{tcolorbox}}\fi

\setstretch{1.5}
\hypertarget{introduction}{%
\section{Introduction}\label{introduction}}

\hypertarget{sec-intro}{%
\subsection{Background}\label{sec-intro}}

Per- and Polyfluoroalkyl substances (PFAS) are a large class of
human-made chemicals that have been manufactured and used in a variety
of industries since the 1940s
\citep{wang_never-ending_2017, gluge_overview_2020, gaines_historical_2023}.
PFAS have been or are currently being synthesized for a myriad of
different uses, including adhesives, stain resistant coatings for
clothes or furniture and fire retardants. In addition to consumer and
industrial applications, PFAS are being released into the environment
during manufacturing and use \citep{wallis_source_2023}. PFAS and
products containing them are regularly disposed of in landfills or
incinerated, which can also lead to further release into soil,
groundwater, and air \citep{chen_evaluation_2023, li_critical_2023}.
They are also found in biosolids from wastewater treatment facilities
which have been spread onto agricultural fields
\citep{bolan_distribution_2021}.

Characterizing the scope and scale of the `PFAS class' has been
challenging in the absence of a harmonized PFAS definition. Some
publications have cited thousands of PFAS being in the environment
(estimates range from 4700 \citep{oecd_reconciling_2021} to greater than
10,000 \citep{gaber_devil_nodate}), but there is likely to be an
increasing number given that analytical methods are continually being
evolved to detect them. An Organisation for Economic Co-operation and
Development (OECD) working group defined PFAS as `fluorinated substances
that contain at least one fully fluorinated methyl or methylene carbon
atom (without any H/Cl/Br/I atom attached to it); that is, any chemical
with at least a perfluorinated methyl group (--CF3) or a perfluorinated
methylene group (--CF2--)' \citep{oecd_reconciling_2021, wang_new_2021}.
This broad OECD definition would make estimates of a few thousand PFAS
too low; however, the OECD working group also acknowledges that a
chemistry definition of PFAS does not necessarily equate to how PFAS
should be assessed in terms of their hazard profile or to what extent
subcategorizations of PFAS are appropriate depending on different
legislative frameworks. Indeed, if the OECD definition were applied to a
large inventory such as the US EPA's Distributed Structure-Searchable
(DSSTox) Database \citep{grulke_epas_2019} estimates of the number of
PFAS would be in the order of 30,000. For contrast, the
\href{https://pubchem.ncbi.nlm.nih.gov/classification/\#hid=120}{PubChem
Classification Browser} has tagged over 7 million substances as meeting
the OECD PFAS definition \citep{schymanski_per-_2023}. This would imply
that any substance containing a CF3 would be classified as a ``PFAS''
even though it might fall within the purview of different regulatory
frameworks. The US EPA's Office of Pollution Prevention and Toxics
(OPPT) recently finalized a structural definition of PFAS applicable to
several Toxic Substances Control Act (TSCA) activities: the Significant
New Use Rule (SNUR) on PFAS designated as inactive on the TSCA inventory
\citep{epa_2023}, the TSCA Report and Recordkeeping Requirements for
Perfluoroalkyl and Polyfluoroalkyl Substances rule (referred to herein
as the TSCA section 8(a)(7) rule) \citep{epa_2023b} and EPA's recently
released framework for assessing PFAS under TSCA's New Chemicals
activities \citep{epa_2023c}. For these TSCA actions, a PFAS is defined
as `including at least one of three substructures: 1) R-(CF2)-CF(R')R'`,
where both the CF2 and CF moieties are saturated carbons; 2)
R-CF2OCF2-R', where R and R' can either be F, O, or saturated carbons;
or 3) CF3C(CF3)R'R'`, where R' and R'\,' can either be F or saturated
carbons. This definition is narrower in scope than the OECD chemistry
definition yet EPA estimates that it would still identify several
thousand PFAS.

Of the many thousands of PFAS, few have been studied extensively in
terms of their toxicity profile. Beyond a handful of closely-related
perfluoroalkyl acids, perfluoroalkyl sulfonates, and perfluoroalkyl
ethers (e.g., perfluoroocanoic acid (PFOA), perfluorooctane sulfonic
acid (PFOS), and hexafluoropropylene oxide dimer acid, HFPO-DA), the
vast majority of PFAS lack data to facilitate a robust characterization
of their potential toxicity \citep{carlson_systematic_2022}. In an
effort to address these data gaps, Congress directed EPA (15 USC 8962)
to develop a process for prioritizing which PFAS or `class' of PFAS
should be subject to additional research efforts based on potential for
human exposure, potential toxicity, and other available information. In
response, the EPA published the
\href{https://www.epa.gov/assessing-and-managing-chemicals-under-tsca/national-pfas-testing-strategy}{EPA
National PFAS Testing Strategy} in October 2021 which describes EPA's
approach to developing categories of PFAS and identifying substances for
further data collection efforts.

The notion of a `class' underpins grouping approaches which includes the
concept of developing categories to perform associated read-across.
Rather than assessing each PFAS individually, closely related PFAS could
be, in principle, grouped together into categories. Thus, in a category
approach, not every PFAS needs to be tested for every single endpoint.
Instead, the overall data for that category could potentially prove
applicable to support a hazard assessment for other members of the
category.

Grouping approaches have been in use in regulatory programmes for many
years dating back to 1998 when guidance was developed by the EPA in
support of the
\href{https://nepis.epa.gov/Exe/ZyPURL.cgi?Dockey=P1004QXK.TXT}{US High
Production Volume (HPV) Challenge Program} \citep{usepa_2004_hpv}. The
concepts of grouping, categories and read-across are defined and
extensively described in OECD's grouping guidance document, last revised
in 2014 \citep{oecd_guidance_2017} and presently undergoing revision.
Moreover, the state of the art in read-across has also been described
extensively in the literature; from workflows which outline the steps
undertaken to develop category and analogue approaches through to the
evaluation, justification and documentation of any read-across
predictions made
\citep{cronin_chapter_2013, escher_towards_2019, patlewicz_navigating_2018, patlewicz_towards_2023}.
More recently the notion of enhancing structure-based groupings with new
approach methods (NAMs) has also been an evolving topic. For example, of
keen interest is the extent to which structural categories can be
further justified by NAM data by providing a mechanistic underpinning
\citep{escher_towards_2019, patlewicz_navigating_2018, patlewicz_towards_2022, patlewicz_towards_2023}.
NAMs are defined as any technology, methodology, approach, or
combination that can provide information on chemical hazard and risk
assessment without the use of animals, including \emph{in silico},
\emph{in chemico}, \emph{in vitro}, and \emph{ex vivo} approaches
\citep{stucki_2022}. Of note, EPA has been leading a research programme
to test a targeted set of \textasciitilde150 PFAS through an array of
different NAM approaches as part of a category approach
\citep{carstens_evaluation_2023, houck_bioactivity_2021, houck_evaluation_2023, kreutz_category-based_2023, patlewicz_towards_2022, smeltz_plasma_2023, smeltz_targeted_2023, stoker_high-throughput_2023, degitz_2024}.

This study describes the approach taken to further refine a relevant
PFAS landscape to EPA from which an initial set of structural categories
were derived. The work here is a continuation of the initial
categorization efforts described in the
\href{https://www.epa.gov/assessing-and-managing-chemicals-under-tsca/national-pfas-testing-strategy}{EPA
National PFAS Testing Strategy}. For the categories developed, data gaps
were assessed to help identify which categories were particularly data
poor (e.g., lacking relevant repeated dose toxicity data) and/or
associated with known exposures and therefore would benefit from data
collection or new test data generation (using both NAMs or traditional
approaches) to better characterize the category as a whole. The aims of
this manuscript are as follows:

\begin{enumerate}
\def\labelenumi{\arabic{enumi}.}
\tightlist
\item
  Summarize the process of constructing a PFAS landscape;
\item
  Profile the PFAS landscape to assign substances into broad structural
  categories in combination with chain length;
\item
  Evaluate the degree of structural similarity within each category and
  determine which categories needed to be further subset to maximize
  their structural similarity whilst maintaining a pragmatic total
  number of categories;
\item
  Facilitate the identification of potential candidate PFAS for data
  collection by capturing additional considerations such as availability
  of a known manufacturer/importer (who would be responsible for
  conducting testing via TSCA); EPA Agency and/or State priorities,
  environmental monitoring information and structural diversity within
  the category;
\item
  Evaluate the categories based on their predicted physical state and
  physicochemical properties (a context of evaluating the similarity
  within the category and informing on potential technical limitations
  for testing);
\item
  Consider the utility of the structural categories developed in
  performing read-across, as well as refinements such as incorporating
  mechanistic and toxicokinetic data derived from NAMs. The mechanistic
  insights derived from EPA's parallel research effort on selected PFAS
  offer potential opportunities to refine the structurally-based
  categories developed.
\item
  Evaluate the feasibility of operationalizing the structural categories
  so that new PFAS can be profiled and assigned into one of the terminal
  categories developed.
\end{enumerate}

\hypertarget{methods}{%
\section{Methods}\label{methods}}

\hypertarget{sec-defining-pfas}{%
\subsection{Defining the PFAS landscape}\label{sec-defining-pfas}}

To define the PFAS landscape for the purpose of this study, the DSSTox
database \citep{grulke_epas_2019, williams_comptox_2017} was searched
using a series of structure-based queries that reflected the PFAS
structural definition described earlier (see Section~\ref{sec-intro}).
DSSTox forms the basis of the EPA CompTox Chemicals Dashboard (referred
to herein as the Dashboard)
\citep{grulke_epas_2019, williams_comptox_2017} and comprises 1,218,248
substances (at the time of writing, May 2024,
https://comptox.epa.gov/dashboard/). As a result of the search, 13,054
substances were identified as forming the initial PFAS landscape for
this study. This landscape is available as a list published on the
Dashboard at
\href{https://comptox.epa.gov/dashboard/chemical-lists/PFAS8a7v3}{PFAS8a7v3}.
This set was then cross referenced with the TSCA inventory (see
Section~\ref{sec-lists}) to identify matches. The TSCA inventory is the
list of chemical substances in commerce (manufactured, processed or
imported) in the US since January 1975 that do not qualify for an
exemption or exclusion under TSCA (TSCA\,inventory; Section\,2.10.1).
Note that EPA maintains two TSCA inventories -- one that is publicly
available and another that contains confidential business information
(CBI) and is not publicly available. Those substances reported to EPA as
in commerce since June 2006 are designated as ``active'' on the TSCA
inventory. For each of the PFAS (active and inactive) listed on the
publicly available TSCA inventory, degradation products were simulated
using the biodegradation model, Catalogic 301C v13.18 within the
commercial software tool, OASIS Catalogic v5.16.1.10 (University As
Zlatarov, Laboratory of Mathematical Chemistry, Bourgas, Bulgaria;
\url{http://oasis-lmc.org/}). The intent was to enrich the landscape for
PFAS likely to be found in the environment that originated from
substances in commerce. The set of PFAS degradation products (2484) for
the parent TSCA substances were added to the initial landscape such that
the final PFAS landscape used in this study comprised 15,538 substances.
Note only degradation products meeting the PFAS definition were
considered. Chemicals were represented by unique DSSTox Substance
Identifiers (DTXSID) \citep{grulke_epas_2019}, Simplified
Molecular-Input-Line-Entry System (SMILES)
(\url{https://www.daylight.com/dayhtml/doc/theory/theory.smiles.html}),
chemical names and CAS Registry Numbers (CASRN). International Chemical
Identifier keys (InChIKeys), (hashed InChI) \citep{heller_inchi_2015}
were used as identifiers for the degradation products. Chemical
substances in the DSSTox database have been curated and standardized to
ensure correctness in chemical structure as well as their associations
to chemical names and other identifiers such as CASRN. Examples of this
curation include checking for errors and mismatches in chemical
structure formats and mapping to identifiers, as well as structure
validation and/or standardization issues such as hyper-valency,
tautomerism, etc \citep{grulke_epas_2019}.

\hypertarget{biodegradation-potential}{%
\subsection{Biodegradation potential}\label{biodegradation-potential}}

Biodegradation predictions were made for PFAS in the landscape that were
on the TSCA inventory using the Catalogic 301C v13.18 model within the
commercial software tool, OASIS Catalogic v5.16.1.10. The biodegradation
Catalogic 301C model simulates aerobic biodegradation under Ministry of
International Trade and Industry, Japan (MITI) I (OECD 301C) test
conditions. The modelled endpoint is the percentage of theoretical
biological oxygen demand (BOD) on day 28. The underlying training set
for the model comprises BOD data for 2618 substances -- 745 of these
were collected from the MITI I database and 804 were provided by
National Institute of Technology and Evaluation (NITE), Japan. A further
1069 substances that were proprietary were provided by NITE, Japan. The
training set includes 797 readily biodegradable and 1821 not readily
biodegradable substances. In addition to BOD data, a second database
underpinning the model comprised pathways for 845 organic substances,
documented pathways for 649 chemicals were collected from the primary
and secondary literature whereas pathways for 196 proprietary substances
were provided by NITE, Japan. In brief, the Catalogic model comprises a
metabolic simulator and an endpoint model. The microbial metabolism is
simulated by a rule-based approach based on a set of hierarchically
ordered transformations and a system of rules controlling the
application of these transformations. Recursive application of the
transformations allows for the simulation of metabolism and generation
of biodegradation pathways. Calculation of the modelled endpoint is
based on the simulated metabolic tree and the material balance of
transformations used to build the tree. Predictions were made for all
PFAS in the landscape that were on the non-confidential TSCA inventory
(see Section~\ref{sec-lists} for more details). Prediction results
containing the list of simulated metabolites (as SMILES) along with
their parent DTXSID identifiers were exported as a text file. Prediction
results were then processed in the following manner:

\begin{enumerate}
\def\labelenumi{\arabic{enumi}.}
\tightlist
\item
  DTXSID identifiers were extracted for each parent substance and mapped
  to each metabolite. This ensured for a given parent, all metabolites
  could be readily associated with its corresponding parent substance.
\item
  A new identifier was then created for the metabolites based on the
  parent DTXSID identifier. That is to say, the first listed metabolite
  simulated for parent DTXSID9065256 would be tagged as
  DTXSID9065256\_m\_1 and so on.
\item
  InChIKeys were then generated for all SMILES, parents and simulated
  metabolites. Use of InChIKeys provided an unambiguous means of
  structurally representing the substance (rather than using SMILES that
  are potentially non-unique) and enabled subsequent associations to be
  derived between substances. The processed results were saved for
  subsequent analysis.
\end{enumerate}

Many degradation products were found to be common across parent
substances. Grouping by InChIKeys created a set of unique degradation
products. These were filtered to remove non PFAS degradation products or
those not meeting the PFAS definition. A final step involved cross
matching the degradates against the starting landscape to remove any
duplicate entries i.e.~if a degradate was a substance already captured.
These steps resulted in a set of 2484 degradation products that were
then added to the starting landscape of 13,054 substances.

To explore the coverage and relevance of the MITI training set (the
non-proprietary portion) within the Catalogic 301C model relative to the
PFAS on the TSCA inventory substances, a comparison was performed to
assess the overlap in structural space as characterized by Morgan
chemical fingerprints \citep{rogers_extended-connectivity_2010} (see
Section~\ref{sec-fingerprints} for details on chemical fingerprint
generation). In the latter case, this structural space was projected
onto a 2-dimensional (2D) scatterplot (see Figure~\ref{fig-miti}) using
a Uniform Manifold Approximation and Projection (UMAP) to facilitate
visualization \citep{mcinnes2020umap}. This is a dimensionality
reduction technique that assumes available data samples are evenly
distributed across a topological space (manifold) which can be
approximated from these finite data samples and mapped to a lower
dimensional space. In essence, UMAP learns the manifold in the high
dimensional space, in this case, these are the 1024 chemical
fingerprints and aims to find a 2D representation of the same manifold.
The Catalogic model also provided an indication of whether any of the
PFAS profiled were part of the training set as well as whether they were
within the structural domain of applicability \citep{dimitrov_2005}.

\hypertarget{profiling-pfas-into-structural-categories}{%
\subsection{Profiling PFAS into structural
categories}\label{profiling-pfas-into-structural-categories}}

This study aimed to develop a hierarchy of PFAS categories starting with
a handful of large, diverse categories that could be subcategorized into
more structurally similar categories based on other considerations
(e.g., chain length and chemical fingerprints). The conceptual workflow
for creating the PFAS structural categories is summarized in
Figure~\ref{fig-categorisation} and the details of each step are
described in turn.

\begin{figure}

{\centering \includegraphics[width=1\textwidth,height=\textheight]{PFAS_categorisation_210524.png}

}

\caption{\label{fig-categorisation}Conceptual workflow for generating
PFAS structural categories}

\end{figure}

\hypertarget{sec-primary-categories}{%
\subsubsection{Primary structural
categories}\label{sec-primary-categories}}

Primary categories were derived by profiling the PFAS landscape of
15,538 substances through the PFAS subgroup classification tool
developed by Su et al. \citep{Su_2024} called
\href{https://www.pfas-atlas.net}{PFAS-Atlas}. As described in Su et al.
\citep{Su_2024}, PFAS can be classified into one of at least four broad
classes:

\begin{itemize}
\tightlist
\item
  Perfluoroalkyl acids (PFAA)
\item
  PFAA precursors
\item
  Polyfluoroalkyl acids
\item
  Other PFAS
\end{itemize}

In practice, when substances are batch processed by PFAS-Atlas, a first
class and second class are assigned. The PFAS-Atlas first class
corresponds to the 4 primary categories described above but additionally
delineates linear substances from cyclic substances to create 8 primary
categories. The tool also designates any substance that is not a PFAS
based by the tool as `Not PFAS'. The PFAS-Atlas second class subdivides
each of the first classes further, for example in the PFAA precursors
categories, hydrofluoroethers would be classified separately from
semi-fluorinated alkenes or perfluoroalkane sulfonyl fluorides (PASFs).
For this study, a hybrid approach was used taking into account
membership size. If the number of substances in a PFAS-Atlas first class
designation exceeded 300 substances, its second class designation was
used. This was performed to limit the number of primary structural
categories, and avoiding large membership sizes such as too many
substances falling into the `Other PFAS' category. The 300 membership
threshold was chosen after manual inspection of the membership counts
following substance assignment into the PFAS-Atlas first and second
class designations. The PFAS-Atlas first class designation was used as
the initial primary category assignment for PFAAs, whereas a handful of
second class assignments were used for PFAA precursors and
Polyfluoroalkyl acids. The PFAS-Atlas second class designation was used
in lieu of ``Other PFAS'' except when the number of substances were very
low. The PFAS-Atlas classification tree is an update of the original
PFAS-Map database framework developed by some of the same authors
\citep{su_database_2021} (and which had been used in the original NTS)
but is more closely aligned with the OECD Terminology 2021 guidance
\citep{oecd_reconciling_2021} with some modifications.

The PFAS landscape was also processed through the OPEn
structure-activity/property Relationship App (OPERA) v2.9 tool
\citep{mansouri_opera_2018} (https://github.com/kmansouri/OPERA) to
derive QSAR-READY SMILES and selected physicochemical property
predictions (as discussed in Section~\ref{sec-physchem}). QSAR-READY
SMILES are standardized SMILES where salts and stereochemistry are
removed. QSAR-READY SMILES were used to facilitate the processing of
substances through PFAS-Atlas. Substances without QSAR-READY SMILES or
which could not be computationally resolved were assigned as
``unclassified''. Substances assigned as ``Not PFAS'' by PFAS-Atlas were
also re-assigned as ``unclassified''.

The category assignments used in this study are captured in
Table~\ref{tbl-primary-cats}.

\hypertarget{tbl-primary-cats}{}
\begin{longtable}[]{@{}
  >{\centering\arraybackslash}p{(\columnwidth - 2\tabcolsep) * \real{0.4545}}
  >{\centering\arraybackslash}p{(\columnwidth - 2\tabcolsep) * \real{0.5455}}@{}}
\caption{\label{tbl-primary-cats}List of PFAS-Atlas class assignments
and the corresponding primary categories used in this study. Si PFAS
refer to Silicon PFAS, HFCs refer to hydrofluorocarbons, PASF-based
substances refer to perfluoroalkane sulfonyl fluorides, PFAA refer to
perfluoroalkyl acids and PolyFACs are polyfluoroalkyl alcohols.
Substances that could not be positively categorized by PFAS-Atlas were
denoted as `other'.}\tabularnewline
\toprule\noalign{}
\begin{minipage}[b]{\linewidth}\centering
PFAS-Atlas first class
\end{minipage} & \begin{minipage}[b]{\linewidth}\centering
Primary category
\end{minipage} \\
\midrule\noalign{}
\endfirsthead
\toprule\noalign{}
\begin{minipage}[b]{\linewidth}\centering
PFAS-Atlas first class
\end{minipage} & \begin{minipage}[b]{\linewidth}\centering
Primary category
\end{minipage} \\
\midrule\noalign{}
\endhead
\bottomrule\noalign{}
\endlastfoot
PFAAs & PFAAs \\
PFAAs, cyclic & PFAAs, cyclic \\
PFAA precursors & PFAA precursors \\
PFAA precursors & PASF-based substances \\
PFAA precursors & n:2 fluorotelomer-based substances \\
PFAA precursors & HFCs \\
PFAA precursors, cyclic & PFAA precursors, cyclic \\
Polyfluoroalkyl acids & Polyfluoroalkyl acids \\
Polyfluoroalkyl acids & PolyFCA derivatives \\
Polyfluoroalkyl acids,cyclic & Polyfluoroalkyl acids,cyclic \\
Other PFAS & Other PFAS \\
Other PFAS & Aromatic PFASs \\
& Si PFASs \\
& Polyfluoroalkanes \\
& others \\
Other PFAS,cyclic & Other PFAS,cyclic \\
Other PFAS,cyclic & others, cyclic \\
Not PFAS & Unclassified \\
\end{longtable}

\hypertarget{secondary-structural-categories}{%
\subsubsection{Secondary structural
categories}\label{secondary-structural-categories}}

It is hypothesized that the length of the contiguous fluorinated carbon
chain influences differences in toxicity as well as the length of time
the chemical spends in the body and environment. This supposition draws
from experiences with PFAAs
\citep{chambers_review_2021, sznajder-katarzynska_review_2019}. Due to
the potential importance of chain length in the toxicity, persistence
and bioaccumulation of PFAS, secondary structural categories were
defined using a carbon chain length threshold.

\emph{Chain length determination}

The maximum number of contiguous CF\textsubscript{2} groups in a chain
was determined for all 15,538 substances. This was achieved by iterating
through a range of CF\textsubscript{2} chain lengths (from 1-30) for
each substance in turn and determining its longest chain length. For
instance, Perfluorosebacamidine {[}DTXSID40380015{]} contains 8
contiguous CF\textsubscript{2} units; hence, its chain length was
denoted as 8. PFOA {[}DTXSID8031865{]} had a maximum chain length of 7
whereas PFOS {[}DTXSID3031864{]} had a maximum chain length of 8. For
PFOA, although there are 8 carbons in its backbone, the 8th is part of
the carboxyl group whereas in PFOS, there are 8 CF\textsubscript{2}
groups plus the sulfonate group.

For the current analysis, the chain length threshold was set at 7
(\textgreater=7 vs \textless7) as representative of a ``long chain''
PFAS. The chain length threshold is broadly consistent with the EPA's
2009
\href{https://www.epa.gov/sites/default/files/2016-01/documents/pfcs_action_plan1230_09.pdf}{PFAS
action plan}. A PFAS with a maximum number of contiguous
CF\textsubscript{2} number greater than or equal to 7 was denoted
``gte7''. Using this threshold, both PFOS and PFOA would be assigned to
the ``gte7'' secondary category. A PFAS with a maximum number of
contiguous CF\textsubscript{2} groups less than 7 was denoted ``lt7''.
Defining chain lengths for PFAS with non-contiguous chains or branching
is less straightforward but has been evaluated in more detail by Richard
et al \citep{richard_identification_2022, richard_new_2023} through the
development of new PFAS specific chemical fingerprints, so-named PFAS
ToxPrints, as an extension of the logic used to develop the original
ToxPrints that had been defined for a broader chemistry
\citep{yang_new_2015}. A secondary category was thus denoted by its
PFAS-Atlas assignment, akin to the primary OECD structural
classification and a carbon chain length threshold e.g.,
2,2,3,3,4,4,5,5,6,6,7,7,8,8,9,9,9-Heptadecafluoro-N,N-diphenylnonanamide
{[}DTXSID90896196{]} would thus be described as belonging to the
``Aromatic PFASs, gte7'' secondary category (see
Figure~\ref{fig-categorisation}).

\hypertarget{sec-term-cat}{%
\subsubsection{Derivation of terminal structural
categories}\label{sec-term-cat}}

The underlying motivation for the study was to identify categories that
would balance maximizing structural similarity that could permit
read-across within those categories versus pragmatism in terms of total
number of categories. Too many categories with very few substances
renders the approach less generalizable, too few categories could result
in extrapolating between substances that were not sufficiently similar.
To that end, an objective threshold was needed to determine how granular
categories needed to be to manage this trade-off and ensure that the
categorization was actionable. An objective threshold was developed,
described in Section~\ref{sec-objective-threshold}, that compared
structural similarity within a category relative to the structural
similarity between different categories.

\hypertarget{sec-fingerprints}{%
\subsubsection{Chemical fingerprints}\label{sec-fingerprints}}

Morgan chemical fingerprints \citep{rogers_extended-connectivity_2010}
were calculated for all substances within each secondary category using
the open-source Python library RDkit \citep{landrum_rdkit}, with a
radius of 3 and a bit-length of 1024. The fingerprint data was
represented as bit vectors where presence of structural features were
denoted by 1 and absence by 0. These fingerprint (FP) files were stored
for subsequent processing. Morgan fingerprints also known as Extended
Connectivity Fingerprints (ECFPs) are widely used in machine learning
applications for cheminformatics \citep{oboyle_comparing_2016}
especially when ranking diverse structures by similarity. These circular
fingerprints map the molecular environment of every atom.

\hypertarget{sec-chem-sim}{%
\subsubsection{Chemical similarity}\label{sec-chem-sim}}

Pairwise distance matrices were calculated for each secondary category.
These were generated by using the chemical fingerprint files as inputs
and computing the Jaccard distance for each pair of substances. The
Jaccard distance captures the proportion of FP bits between 2 substances
that differ\citep{raymond_comparison_2003}. The Jaccard distance ranges
from 0 to 1 where 0 would indicate zero distance (or high similarity)
and 1 would indicate high distance (or low similarity). Distance
matrices were computed for all secondary categories and stored for
subsequent processing. These are referred to as `within category'
distance matrices in Section~\ref{sec-objective-threshold}.

\hypertarget{sec-objective-threshold}{%
\subsubsection{Objective distance
threshold}\label{sec-objective-threshold}}

The rationale underpinning the objective distance threshold was based on
the expectation that the variance in the distribution of the pairwise
distances for each secondary category representing the `within category'
similarity would be lower than distributions of the pairwise distances
between different secondary categories (`between category'). The `within
category' distances had already been computed as described in
Section~\ref{sec-chem-sim}.

`Between category' combinations aimed to identify categories that did
not share the same primary category root. A list of all possible binary
combinations of secondary categories was created using the names of the
secondary categories, ``Aromatic PFASs, lt7'' and ``PFAA precursors,
gte7'' is an example of such a binary combination. These were then
filtered to remove secondary categories that shared the same primary
category root (i.e., a combination such as ``PFAAs, lt7'' and ``PFAAs,
gte7'' would be excluded from consideration as a `between category').
Chemical fingerprint datasets for each binary combination were created
by combining the secondary category chemical fingerprint datasets.
Pairwise distance matrices were then derived for the combined category
set. These matrices were filtered to retain only the pairwise distances
between the starting secondary categories.

The empirical cumulative distribution functions (ECDFs) of the pairwise
distances were calculated for each secondary category (see
Figure~\ref{fig-ecdfs-within} for a plot of the ECDFs). ECDFs were also
derived for the `between category' combinations (see
Figure~\ref{fig-ecdfs-bet} for a plot of the first 10 ECDFs). The ECDFs
permitted a visual inspection of the range of the pairwise distances
across all secondary categories as well as across all `between category'
combinations. Based on visual inspection of the ECDFs, the median value
for each distribution was selected as the summary metric.

Probability density functions of median values from all within and
between secondary categories were plotted to explore their overlap. The
15\textsuperscript{th} percentile of the `between categories'
distribution was selected, by reference to the density plot, as the
threshold to determine whether a secondary category merited further
subcategorization. A secondary category was only subcategorized if the
median of its `within category' pairwise distance distribution exceeded
this threshold.

\hypertarget{sec-terminal}{%
\subsubsection{Deriving terminal categories}\label{sec-terminal}}

Secondary categories that exceeded the threshold were subcategorized
using agglomerative hierarchical clustering. The condensed form of the
pairwise distance matrix computed for each secondary category that
exceeded the threshold was used as an input into a hierarchical
clustering using Ward's method \citep{ward_hierarchical_1963}. Ward's
method is a criterion that minimizes the total within-cluster variance.
For each secondary category, the dendrogram was plotted and the number
of first-generation clusters was set as the maximum cluster number.
Clusters were labelled as 1,2,3 etc. Each of the clustering results were
combined into one table which was then merged with the starting table of
primary and secondary categories.

The next generation of categories, quaternary categories, would then be
processed in the same manner to determine whether any exceeded the
objective threshold and needed to be subcategorized further as already
described. In practice, a maximum of two generations of
subcategorizations were performed, with the expectation that this would
balance the structural similarity within the category relative to total
number of terminal categories.

Secondary categories or tertiary categories which did not exceed the
threshold were ultimately denoted as the terminal category. Thus, a
terminal category could be tagged as ``Aromatic PFASs, gte7'',
effectively a secondary category, or could be tagged as ``Aromatic
PFASs, lt7, 3'', a tertiary category or following two iterations of
subcategorization would be tagged as ``Aromatic PFASs, lt7, 4, 1'' (see
Figure~\ref{fig-categorisation}). Note: in the data files and figures,
terminal categories without one or two iterations of subcategorization
are denoted as ``Aromatic PFASs, gte7, nan, nan'' or ``Aromatic PFASs,
lt7, 3, nan'' where ``nan'' represents a null value.

\hypertarget{sec-centroid}{%
\subsubsection{Identification of centroid
substances}\label{sec-centroid}}

For each terminal category, a single substance was identified that was
nominally representative of the category. This substance was the
computed centroid calculated from the Jaccard pairwise distance matrices
(see Section~\ref{sec-chem-sim}). The sum of the pairwise distances
across all substances for a given structural category was computed and
the substance with the minimum value was denoted as the centroid (i.e.,
this substance would have the lowest distance from all other category
members). Technically, this calculation gives rise to the medoid of a
cluster. However, for the purposes of this analysis and for consistency
with the NTS, the term centroid is used to denote it as the `central'
substance within the category. Distances of all category members
relative to the centroid substance were also computed.

\hypertarget{sec-maxmin}{%
\subsubsection{Identification of additional representative
substances}\label{sec-maxmin}}

Since a number of the terminal categories were large in size
(e.g.~greater than 300 members), a single substance would be potentially
insufficient to both characterize the category and its potential hazard
profile. In an effort to address this limitation, the MaxMinPicker
approach, as implemented within the RDKit Python library, was applied to
identify additional substances which would in turn capture the breadth
and diversity of each terminal category
\citep{ashton_identification_2002}. The MaxMin approach is a
well-established algorithm for dissimilarity-based compound selection
that has been applied in drug discovery for many years. The reader is
referred to Snarey et al \citep{snarey_comparison_1997} for a comparison
of the different algorithms. The MaxMinPicker approach proceeds as
follows:

1. Molecular descriptors are generated for all substances. In this case,
the Morgan fingerprints calculated for all substances within a terminal
category represented the candidate pool whereas the pre-computed
centroid equated to the initial seed.

2. From the substances in the terminal category, the substance that had
the maximum value for its minimum distance to the picked set (initially
this would be just the centroid) would then be identified. This
substance would be the most distant one to those already picked so it
would be transferred to the `picked set' (now centroid + 1).

3. An iteration back to step 2 would then be performed until the desired
number of substances were picked.

The MaxMinPicker was applied to all terminal categories containing more
than 5 members to identify the next 3 most diverse substances within a
category (centroid + up to 3 additional substances). The intention of
identifying additional diverse substances was to help bound the domain
of the structural category. The identification of 3 most diverse
substances was chosen out of convenience to provide an actionable number
of additional substances.

A systematic evaluation of the relationship between the number of
diverse substances that could be identified relative to the structural
diversity within each terminal category was also undertaken. This was
performed as follows, first the ranked order by diversity of all members
within a terminal category was computed. Then the pairwise distance
matrices derived in Section~\ref{sec-chem-sim} were filtered by the
diverse substances, starting from the centroid, centroid plus first
diverse chemical through to the complete set of category members. At
each step the mean minimum distance was recorded. This enabled the
construction of a matrix to capture the mean of the minimum pairwise
distances relative to the number of diverse chemicals selected. The
normalized cumulative sum of all the mean minimum distances was then
computed. This provided a means of evaluating the proportion of
structural diversity that was captured as a function of the number of
MaxMin substances identified.

This calculation provides for two objective assessments namely:

1. the amount of structural diversity captured by the 3 diverse picks
originally identified; and

2. the number of diverse substances that would need to be identified (if
practical resources were not a limiting factor) to capture a specified
level of structural diversity. For example, how many substances would
need to be identified if capturing a specific percentage of the
structural diversity within a terminal category was desired, 80\% is
presented here merely for illustrative purposes.

\hypertarget{facilitating-the-identification-of-potential-candidates-for-data-collection}{%
\subsection{Facilitating the identification of potential candidates for
data
collection}\label{facilitating-the-identification-of-potential-candidates-for-data-collection}}

To facilitate the identification of potential candidate PFAS for data
collection, availability of a known manufacturer/importer, EPA Agency
and/or State priorities, environmental monitoring information were
evaluated as additional considerations. These are described in turn.

\hypertarget{sec-lists}{%
\subsubsection{Qualitative exposure and release
designations}\label{sec-lists}}

Several qualitative designations were added to the landscape to identify
substances for which exposure could be plausible, including their TSCA
inventory status, production volumes per TSCA's Chemical Data Reporting
(CDR) rule, State/EPA Region priorities, as well as physical state and
physicochemical properties.

The non-confidential (Non-CBI) TSCA Inventory active and inactive lists
were downloaded from the Dashboard (see
\href{https://comptox.epa.gov/dashboard/chemical-lists/TSCA_ACTIVE_NCTI_0224}{TSCA\_ACTIVE\_NCTI\_0224},\href{https://comptox.epa.gov/dashboard/chemical-lists/TSCA_INACTIVE_NCTI_0224}{TSCA\_INACTIVE\_NCTI\_0224})
and combined into one large set. Substances within this inventory
included both Chemical Abstract Service (CAS) Registry Number, Chemical
Abstracts (CA) Index Name, and DSSTox substance identifier (DTXSID).
These were matched by the DTXSID identifiers already captured in the
PFAS landscape. Substances were tagged as `inactive', `active' or
`unclassified'. Note the predicted degradation products of substances
tagged as either ``inactive'' or ``active'' had been used to augment the
PFAS landscape as already described in Section~\ref{sec-defining-pfas}.

The 2020 CDR data was downloaded from the public EPA web address
(https://www.epa.gov/chemical-data-reporting/access-cdr-data). The CDR
data comprises information for a set of 8660 substances. DTXSID
identifiers were available for 8017 of these substances when using the
Batch search functionality within the Dashboard. A tag was created for
CDR2020 status if a PFAS had a 2020 CDR record. National Aggregated
Production Volume (National Agg PV) data was also extracted to highlight
how this was distributed across primary categories. Since some of the
production volume (PV) data was numeric and some represented in numeric
ranges, the PV data was summarized into one of 10 different ranges
(\textless25,000 lbs, 25,000-\textless100,000 lbs,
100,000-\textless500,000 lbs, 500,000-\textless1,000,000 lbs,
\textless1,000,000 lbs, 1,000,000 -\textless10,000,000 lbs,
1,000,000-\textless{} 20,000,000 lbs, 20,000,000-\textless{} 100,000,000
lbs, 50,000,000-\textless100,000,000 lbs, 100,000,000-\textless{}
1,000,000,000 lbs).\\

Various EPA Regions or States have identified PFAS of interest based on
validated analytical methods or for environmental monitoring purposes.
The data sources captured as part of the EPA's PFAS Analytic Tools
website (\url{https://echo.epa.gov/trends/pfas-tools\#data}) were used
to construct lists of such PFAS. The specific data sources were
Discharge Monitoring Data, Drinking Water (State) Data, Drinking Water
(Unregulated Contaminant Monitoring Rule (UCMR)) Data, Environmental
Media Data, Production Data, Toxics Release Inventory (TRI) Data --
Waste Managed, TRI Data -- On-Site, TRI Data -- Off-Site and Production
Data (all accessed 7th April 2024). Discharge Monitoring data is
collected by virtue of the National Pollutant Elimination System permit.
Drinking Water Data comprises UCMR and State level monitoring data.
Environmental Media data comprises ambient sampling data reported by
federal, state, tribal and local governments, academic and
non-governmental organizations, and individuals that are submitted to
the Water Quality Portal (WQP). Production data entails information
reported under the Chemical Data Reporting (CDR) Rule under TSCA. TRI
tracks the management of certain toxic chemicals that may pose a threat
to human health or the environment by more than 21,000 facilities
throughout the US and its territories. The National Defense
Authorization Act of Fiscal year 2020 (NDAA) added certain PFAS to the
TRI list and provided a framework for the ongoing listing of additional
PFAS.

Identifiers were extracted from these source files and searched against
the Dashboard to map to DTXSID records. The set of identifiers (Names
and CASRN) within the entire PFAS landscape were also queried against
PubMed, the National Library of Medicine's citation index for biomedical
literature, to determine whether studies for a substance might have been
reported in the literature. The article counts were obtained using the
same queries as used within the Abstract Sifter v7.5
\citep{baker_abstract_2017}.

Expected routes of exposure and presence in environmental media are
dependent on the physical state and physicochemical properties.
Physicochemical properties were predicted using the open-source OPERA
v2.9 tool \citep{mansouri_opera_2018} for all substances with QSAR-READY
SMILES (as discussed in Section~\ref{sec-primary-categories}). The
properties predicted were melting point, boiling point, Henry's Law
constant, water solubility and vapour pressure. Physical state was
predicted at 25 deg C using the predicted values of melting point and
boiling point. Gases had boiling points less than 25 deg C, solids had
melting points greater than or equal to 25 deg C and liquids had melting
points less than 25 deg C and boiling points greater than or equal to 25
deg C. These are the guiding principles underpinning the EPA's
Sustainable Futures Framework guidance (see
\href{https://www.epa.gov/sites/default/files/2015-05/documents/05-iad_discretes_june2013.pdf}{Interpretative
Guidance Document}). Whilst physical properties are continuously
distributed, and cutoff values are necessarily arbitrary, there is
utility in grouping substances into broad categories as a way to
acknowledge the practicalities of testing and human exposure under
``typical'' (i.e.~room temperature and atmospheric pressure) conditions.
A water solubility threshold of 0.5 mg/L was used to denote whether a
substance was soluble/insoluble whereas a vapour pressure threshold of
75 mmHg determined volatility and a HLC threshold of 0.1 atm m3/mol
highly volatile. Based on these properties, each substance was assigned
into 1 of 4 ``physical state and physicochemical designations'' (from
A-D). Designation A covered substances that were insoluble solids,
designation B captured both soluble solids and soluble non-volatile
liquids, whereas C tagged soluble volatile liquids/insoluble liquids and
soluble gases. Designation D assigned substances as insoluble gases or
highly volatile gases. Substances that could not be assigned into one of
these 4 designations were tagged as `not determined'. For each of the
terminal structural categories, Morgan fingerprint representations were
projected into two dimensions using UMAP to facilitate visualization
\citep{mcinnes2020umap}. The projections were plotted as 2D kernel
density distributions overlaid with physical state and physicochemical
designation information to help explore the extent to which members were
assigned to the same designation and therefore had a consistent profile
across a given terminal category.

Each of these respective qualitative designations were then matched to
the PFAS landscape to provide another attribute for consideration when
identifying potential candidates for data collection.

\hypertarget{sec-constrained}{%
\subsubsection{Constrained PFAS landscape}\label{sec-constrained}}

One of the limitations of the identification of centroids and additional
diverse substances was that they might yet not yield feasible candidates
for data collection due to the lack of assignable manufacturer/importer.
This was articulated as a potential challenge in the National PFAS
Testing Strategy. To address this practical constraint, the same process
of computing centroids, identifying additional diverse substances and
evaluating their structural diversity coverage was also performed using
the terminal categories as a basis as described in
Section~\ref{sec-terminal} but constraining the landscape to only those
substances on the public TSCA inventory and specifically those
substances that were actives on the public TSCA inventory. Constraining
the landscape would allow identification of substances for data
collection that were already in commerce and/or could be more readily
procured.

\hypertarget{sec-invivo}{%
\subsection{\texorpdfstring{Evaluation of variance of \emph{in vivo}
toxicity within terminal
categories}{Evaluation of variance of in vivo toxicity within terminal categories}}\label{sec-invivo}}

Ultimately, read-across of data within categories could be performed
such that the hazard profile of the category is adequate without needing
to test a significant number of category members. To evaluate the
feasibility of performing read-across within the terminal categories
derived, an exploration of the distribution of \emph{in vivo} points of
departure (PODs) within and across terminal categories was performed for
the oral route of exposure.

\hypertarget{sec-rax}{%
\subsubsection{\texorpdfstring{Variance of \emph{in vivo} PODs across
and within terminal
categories}{Variance of in vivo PODs across and within terminal categories}}\label{sec-rax}}

From ToxValDB version 9.5, the
\href{https://www.epa.gov/comptox-tools/downloadable-computational-toxicology-data\#AT}{Toxicity
Values Database}, all studies where `oral' was the route of exposure
were extracted. Only records where a point of departure (POD) was
reported as a NOEL, NOAEL, NEL, NOAEC, LOAEL, LOEL, LOAEC, LEL and where
the dose units were expressed as mg/kg-bw/day or mg/kg were retrieved.
Study types were also restricted to the following: `short-term',
`subchronic', `chronic', `developmental', `reproduction', `reproduction
developmental', `28-day' as captured in the `study type' field within
the database. Species were standardized into one of `rat', `mouse',
`rabbit', `dog', `hamster' or `guinea pig'. Effect levels were
harmonized consistent with the approach taken by Aurisano et al.
\citep{aurisano_2023} where non-cancer effects
vs.~reproductive/developmental effects were processed separately.
Records with sub-acute or sub-chronic as the study type were
extrapolated to chronic using a subchronic-to-chronic factor of 2 and a
subacute-to-chronic factor of 5\footnote{Note: the approach taken in
  Aurisano et al.~is not necessarily consistent with uncertainty factor
  selection information provided in EPA's 2002 Review of the Reference
  Dose and Reference Concentration Processes and EPA's 2022 ORD Staff
  Handbook for Developing IRIS Assessments}. LOAEL effect level types
were extrapolated to NOAELs by dividing by an extrapolation factor of 3.
Effect levels for all records were extrapolated to humans by dividing
reported effect values by conversion factors based on the average body
of weight of humans relative to the average body weight of the test
species. NOAELs were extrapolated to human equivalent Benchmark Dose
(BMDh) values based on assigned conceptual model depending on the
critical effect reported. The calculated BMDh was based on the mean of 2
assigned conceptual models. In Aurisano et al \citep{aurisano_2023}, the
25\textsuperscript{th} percentile of the fitted log-normal distributed
(using the mean BMD and standard deviation (sd) BMD) was calculated to
derive a POD per substance. The sd was set to the median sd of all
records in cases where the number of study records was less than 5.

The derived BMDh values were then merged with the PFAS substances from
the landscape. The summary values provided an estimate of the POD for
each substance and the expected level of variation across and within
categories. Box and whisker plots were created to reflect the
distribution of the PODs across the terminal categories for general
non-cancer and repro/developmental effects for the oral route of
exposure. Strip plots were overlaid to show the variation of chain
length across a given terminal category for general non-cancer effects.

\hypertarget{sec-nam-flag}{%
\subsection{Qualitative mechanistic and toxicokinetic
designations}\label{sec-nam-flag}}

A summary of the NAM testing being undertaken for \textasciitilde150
PFAS was described in Patlewicz et al. \citep{patlewicz_towards_2022}.
See Houck et al. \citep{houck_bioactivity_2021} for results from various
nuclear receptor and oxidative stress targeted assays, Houck et al.
\citep{houck_evaluation_2023} for 12 human primary cell-based assay
models of pathophysiology including immunosuppression, Carstens et al.
\citep{carstens_evaluation_2023} for the developmental neurotoxicity
assays, Degitz et al. \citep{degitz_2024} for the thyroid pathway assays
and, for toxicokinetic information, Smeltz et al.
\citep{smeltz_plasma_2023} and Kreutz et al.
\citep{kreutz_category-based_2023}. The manuscript for the remaining
data stream (zebrafish developmental toxicity) is in under internal
review (Britton et al., \emph{in prep}).

In addition to the NAM testing, a quality control (QC) evaluation of the
chemical stock solutions was undertaken to confirm PFAS analyte presence
and stability \citep{smeltz_targeted_2023}. This evaluation was
warranted given recent reports of certain PFAS degrading in the aprotic
solvent dimethyl sulfoxide (DMSO), readily used as the solvent of choice
in HTS \citep{liberatore_solvent_2020, zhang_stability_2022}. Two
hundred and five PFAS selected based on criteria described in Patlewicz
et al. \citep{patlewicz_towards_2022} were evaluated using low
resolution tandem mass spectrometric detection strategies to confirm
presence of intended analyte, evaluate analyte stability and presence of
isomers, and verify stock concentrations for a subset for which
commercially available verified standards were available. Ultimately 57
PFAS failed QC evaluation, with three exhibiting degradation in DMSO and
the remainder not detected as present, likely due to volatilization. The
pass/fail scores and informational flags as described in Smeltz et al.
\citep{smeltz_targeted_2023}, and can be downloaded from the following
figshare url,
\url{https://epa.figshare.com/articles/dataset/Chemistry_Dashboard_Data_Analytical_QC_for_PFAS/22118099}.

For each of the NAM data streams, substances were tagged with a
qualitative flag to indicate the class of mechanistic information that
could be derived from the associated assay outcome (e.g., estrogen
receptor activity from a nuclear receptor assay) and an expert-derived
qualitative level of confidence associated with the outcome (high
confidence of activity, medium confidence of activity or low concern).
Only NAM results from substances that passed QC were carried forward.
These flags were considered as an additional line of evidence to
determine whether a terminal category might merit being split based on
its mechanistic or toxicokinetic information or to inform what types of
higher order testing might be most impactful for a given substance drawn
from said terminal category. The derivation of the flags are described
in more detail in Judson et al.~\emph{in prep}. Confidence scores across
the NAM flags were standardized as appropriate to facilitate
visualizations across data streams. Each flag could take on one of three
values, low concern, medium or high confidence, color coded as blue,
yellow and red. The immune flag was the exception. It was binary in
nature and only gave rise to a low concern and medium confidence value.
The flag categories are summarized below in Table~\ref{tbl-nam}. One
final parameter computed was a TK half-life bin score using the machine
learning model developed by Dawson et al. \citep{dawson_2023}.
Predictions were scored from 1-4 where 1 signified a half-life ≤ 12 h, 2
= 12 h−1 week, 3 = 1 week−2 months, and 4 ≥ 2 months. Predictions were
generated for humans assuming a oral dosing regimen and aggregated by
maximum half-life score so that there was a single prediction per
substance.

\hypertarget{tbl-nam}{}
\begin{longtable}[]{@{}
  >{\raggedright\arraybackslash}p{(\columnwidth - 6\tabcolsep) * \real{0.1000}}
  >{\centering\arraybackslash}p{(\columnwidth - 6\tabcolsep) * \real{0.3000}}
  >{\centering\arraybackslash}p{(\columnwidth - 6\tabcolsep) * \real{0.3000}}
  >{\centering\arraybackslash}p{(\columnwidth - 6\tabcolsep) * \real{0.3000}}@{}}
\caption{\label{tbl-nam}Summary of NAM Flag Rationales}\tabularnewline
\toprule\noalign{}
\begin{minipage}[b]{\linewidth}\raggedright
Endpoint
\end{minipage} & \begin{minipage}[b]{\linewidth}\centering
Low Concern (Blue)
\end{minipage} & \begin{minipage}[b]{\linewidth}\centering
Medium Confidence (Yellow)
\end{minipage} & \begin{minipage}[b]{\linewidth}\centering
High Confidence (Red)
\end{minipage} \\
\midrule\noalign{}
\endfirsthead
\toprule\noalign{}
\begin{minipage}[b]{\linewidth}\raggedright
Endpoint
\end{minipage} & \begin{minipage}[b]{\linewidth}\centering
Low Concern (Blue)
\end{minipage} & \begin{minipage}[b]{\linewidth}\centering
Medium Confidence (Yellow)
\end{minipage} & \begin{minipage}[b]{\linewidth}\centering
High Confidence (Red)
\end{minipage} \\
\midrule\noalign{}
\endhead
\bottomrule\noalign{}
\endlastfoot
Nuclear Receptors & No nuclear receptor activity & Activity against at
least one of the receptors ER, PPARA, PPARG, PPARD, NFE2L2, PXR, RARG,
RXRB at the level of one or more samples in one assay. & Activity in the
yellow medium concern that is confirmed in at least one sample in 2
orthogonal assays \\
DNT & No activity or activity was only observed at the highest
concentration related to cytotoxicity & Low number of hits which
demonstrated selective bioactivity & Moderate to high bioactivity (as
measured by hitcall) and demonstrated selective bioactivity (activity
below cytotoxicity AC50 as measured by AUC) and median AC50 \textless{}
10 µM \\
Zebrafish & Development was normal in all larvae & Test results were
equivocal or if less than 50\% of the larvae were affected & Positive
activity (i.e., elicited death, non-hatching, or malformations in at
least 50\% of the animals) \\
Thyroid & No activity greater than 50\% of the model inhibitors/binders
& Activity greater than 50\% of the model inhibitors/binder, but the
concentration necessary to result in this activity was 2 orders of
magnitude higher than the model inhibitors/binders & AC50s that were
within 2 orders of magnitude of the model inhibitors/binders \\
Immune & Selectivity scores less than 0.25 log10 µM & Selectivity scores
of greater than 0.25 log10 µM & \\
TK Plasma Binding (TK\_PlasBind) & TK\_PlasBInd\_High: Plasma protein
binding higher than 50\% of non-PFAS chemicals (f\_up \textless{} 0.11)
(this corresponds to 25\textsuperscript{th} percentile of PFAS
(fup\textless0.10) & TK\_PlasBInd\_Higher: Plasma protein binding higher
than 50\% of PFAS chemicals (f\_up \textless{} 0.0109) &
TK\_PlasBInd\_Highest: Plasma protein binding higher than 75\% of PFAS
(f\_up \textless{} 0.0039) \\
TK Instrinic Clearance (TK\_Metab) & TK\_Metab\_Moderate: Clint in upper
75\textsuperscript{th} percentile of exp PFAS data
(Clint\textgreater5.97 ul/min/million cells). Max Clint = 49.86 &
TK\_Metab\_Slow: Clint\textless5.97 ul/min/million heps. (lower
75\textsuperscript{th} percentile) & TK\_Metab\_Stable: Stable in
\emph{in vitro} heptatocyte incubation (Clint = 0 or Clint pvalue
\textgreater{} 0.05) \\
TK\_Struc\_Endo & & Non-fluorinated structure is similar to endogenous
chemicals. More likely to be a transporter substrate. & \\
TK half-life predictions & category 1 or 2 to denote half-life ≤ 12 h,
or 12 h−1 week & category 3 = 1 week−2 months & category 4 ≥ 2 months \\
\end{longtable}

Qualitative observations of the consistency of the various flags across
all the tested substances and within terminal categories were made. The
Fisher‛s exact test was used to compute an odds ratio and associated p
value for each PFAS ToxPrint \citep{richard_new_2023} relative to a NAM
flag that had been converted into a binary scale. This enrichment
analysis was comparable with the methodology discussed in Wang et
al.\citep{wang_high-throughput_2019}. A PFAS ToxPrint was considered
enriched if it had an odds ratio greater than or equal to 3, an
one-sided Fishers exact p-value less than 0.05 (probability value of the
odds ratio being greater than 1) and the number of true positives equal
or greater than 3. The set of `enriched' PFAS ToxPrints were then used
to profile the entire PFAS landscape to assign potential predicted NAM
flags. Comparisons were made of the actual NAM flags and their
predictions to evaluate performance metrics (sensitivity and
specificity). TK half-life predictions were generated for all substances
and the outcomes binarized where substances with a bin category of 4
were assigned a 1, and any other bin category lower was assigned a 0.

\hypertarget{operationalizing-the-terminal-categories-for-re-use}{%
\subsection{Operationalizing the terminal categories for
re-use}\label{operationalizing-the-terminal-categories-for-re-use}}

In the absence of a model to predict the terminal category, the
categorization would need to be re-run for each new set of PFAS. To
operationalize the terminal categories for practical use, a machine
learning approach was used to develop a model that could be used to
profile a new PFAS and assign it to its most likely terminal category. A
random forest classifier (RFC) as implemented in the python library
scikit-learn \citep{pedregosa2011} was used to predict assignment of
substances into one of the final terminal categories developed. Morgan
fingerprints generated earlier (as discussed in
Section~\ref{sec-fingerprints}) in conjunction with primary category and
chain length were combined with the final terminal category names for
all substances. Terminal categories with less than 10 members were
aggregated together into one miscellaneous category. The dataset was
then split into a training 80\% and test 20\% split using a random
stratification approach based on the terminal category labels. A dummy
classifier was applied first to establish a baseline. Then an initial
RFC with default settings was assessed within a 5-fold stratified cross
validation (CV) procedure to evaluate initial performance using balanced
accuracy as a metric. This was performed using a pipeline within
scikit-learn where the primary category names were treated as category
features and passed into an OrdinalEncoder whereas chain length and
Morgan fingerprints were first standardized before being passed to the
RFC. A randomized search was then undertaken as part of a nested 5-fold
CV to identify the best parameters and evaluate test CV performance. The
resulting model was then applied to the test set that had been held out
to evaluate performance. Finally, the model was refitted to the entire
dataset.

\hypertarget{sec-code}{%
\section{Data analysis software and code}\label{sec-code}}

Data processing was conducted using the Anaconda distribution of Python
3.9 and associated libraries. Jupyter Notebooks, scripts and datasets
will be made available on github and Figshare.

\hypertarget{results-and-discussion}{%
\section{Results and discussion}\label{results-and-discussion}}

\hypertarget{primary-and-secondary-structural-categories}{%
\subsection{Primary and secondary structural
categories}\label{primary-and-secondary-structural-categories}}

The PFAS landscape following application of the TSCA section 8(a)(7)
rule to DSSTox resulted in a dataset comprising 13,054 substances plus
2484 degradation products for a total of 15,538.

Minimal structural overlap was found between the 1549 accessible
training set substances (of which 1429 substances could be resolved into
structures) from the Catalogic model and the TSCA inventory substances.
Figure~\ref{fig-miti} depicts a UMAP plot for the MITI training set
substances relative to the TSCA substances using Morgan chemical
fingerprints as inputs. There were 12 TSCA substances of the dataset
that were part of the training set. Only 29\% of the PFAS TSCA
substances were tagged as being within the structural domain of the
model. In view of this, the degradation products simulated (comprising
16\% of the landscape) should be interpreted with caution until
additional experimental data are collected and new models developed.

A chain length could not be computed for 14 substances due to issues
with resolving structures within RDKit. Only one of the 14 substances,
DTXSID20153820, could be resolved since the chain length failed on
account of the chain length exceeding the range used to calculate the
maximum values. The chain length of this substance was manually
annotated. The remaining 13 substances were dropped from further
consideration. There were 31 substances that were tagged as ``Not PFAS''
based on the OECD structure definitions used within PFAS-Atlas. All were
reassigned to the ``unclassified'' primary category.
Figure~\ref{fig-secondary-cats} is a bar chart showing the number of
PFAS within each secondary category. The final PFAS landscape used for
the remainder of the analysis comprised 15,525 substances.

\begin{figure}

{\centering \includegraphics{nts_files/figure-pdf/fig-secondary-cats-output-1.pdf}

}

\caption{\label{fig-secondary-cats}Bar chart showing the number of
substances within each secondary category, ordered by primary category
root. Methods to define primary and secondary categories are outlined in
Sections 2.31 and 2.32. Lt7, chain length less than 7; gte7, chain
length greater than or equal to 7.}

\end{figure}

Across the more than 15,000 PFAS substances evaluated, twenty-two
percent (3430) of the substances fell into the ``Aromatics PFAS, lt7''
category. In addition, 1066 substances fell into the ``Others, lt7'' and
1180 in the ``PFAA precursors, lt7'' secondary categories. This
represents a potential limitation of using broad definitions represented
by the OECD primary categories themselves. The smallest secondary
category was the ``Others PFAS, cyclic, gte7'' with 14 members whereas
``PFAAs, cyclic, gte7'' was a singleton.

A chemotype ToxPrint enrichment was explored following the approach
outlined in Wang et al.\citep{wang_high-throughput_2019} but using the
PFAS specific ToxPrints developed in Richard et al
\citep{richard_new_2023} (see Section~\ref{sec-supp} for methodological
details). This was an effort to identify whether there were specific
structural features that might be helpful in subcategorizing those
primary categories with the largest memberships namely (i.e., ``Aromatic
PFASs'', ``PFAA precursors'' or ``unclassified''). The most enriched
features for the ``unclassified'' category included fluorotelomer chains
and sulfonic acid functional groups whereas alcohols and carbonyls
featured as functional groups for the ``PFAA precursors''. No specific
features were enriched for the ``Aromatic PFAAs'' categories. However,
where there were enriched features, these were not determined to be
sufficiently distinctive to justify creation of additional primary
categories.

Structural similarity was evaluated within and between secondary
categories to determine which secondary categories required further
subcategorization (as discussed in Section~\ref{sec-objective-threshold}
of the Methods). Figure~\ref{fig-threshold} shows the two distributions
of the median pairwise distance distributions in the between and within
secondary category combinations. The objective distance threshold
derived by taking the 15\textsuperscript{th} percentile of the median
pairwise distances from the between categories combinations resulted in
a value of 0.8.

\begin{figure}

{\centering \includegraphics[width=0.5\textwidth,height=\textheight]{Figure2_080524.png}

}

\caption{\label{fig-threshold}Probability density functions of the
median Jaccard pairwise distance distributions for within (orange) and
between (blue) secondary categories. Orange and blue graphed lines
represent the fits to the probability density distributions.}

\end{figure}

Based on the threshold, 16 secondary categories
(Table~\ref{tbl-require-cat}) were found to exceed the value that would
render them subject to further subcategorization. The sixteen secondary
categories included the ``Aromatic PFASs'', ``PFAA precursors'',
``Others'' and ``PolyFCA derivatives''. These categories are of little
surprise given their membership sizes were the largest out of all the
secondary combinations; hence, these categories were expected to be the
most diverse in terms of their structural makeup. Since the ``PFAAs,
cyclic'' category only comprised 1 substance, this was excluded from any
subcategorization.

\hypertarget{tbl-require-cat}{}
\begin{longtable}[]{@{}
  >{\raggedright\arraybackslash}p{(\columnwidth - 2\tabcolsep) * \real{0.4583}}
  >{\raggedright\arraybackslash}p{(\columnwidth - 2\tabcolsep) * \real{0.5417}}@{}}
\caption{\label{tbl-require-cat}List of secondary categories exceeding
the threshold and their corresponding median pairwise distances (rounded
to 2 decimal places)}\tabularnewline
\toprule\noalign{}
\begin{minipage}[b]{\linewidth}\raggedright
Primary-Secondary Categories
\end{minipage} & \begin{minipage}[b]{\linewidth}\raggedright
Median Within-Category Pairwise distance
\end{minipage} \\
\midrule\noalign{}
\endfirsthead
\toprule\noalign{}
\begin{minipage}[b]{\linewidth}\raggedright
Primary-Secondary Categories
\end{minipage} & \begin{minipage}[b]{\linewidth}\raggedright
Median Within-Category Pairwise distance
\end{minipage} \\
\midrule\noalign{}
\endhead
\bottomrule\noalign{}
\endlastfoot
others, cyclic, lt7 & 0.92 \\
PFAA precursors, cyclic, lt7 & 0.90 \\
Other PFASs, cyclic, lt7 & 0.89 \\
Aromatic PFASs, lt7 & 0.88 \\
Other PFASs, lt7 & 0.88 \\
unclassified, lt7 & 0.87 \\
others, lt7 & 0.87 \\
‍Polyfluoroalkyl acids, cyclic, lt7 & 0.86 \\
PFAA precursors, lt7 & 0.86 \\
Other PFASs, cyclic, gte7 & 0.85 \\
‍Polyfluoroalkanes, lt7 & 0.85 \\
‍PolyFCA derivatives, lt7 & 0.83 \\
‍PFAAs, lt7 & 0.82 \\
PFAAs, cyclic, lt7 & 0.82 \\
‍Polyfluoroalkyl acids, lt7 & 0.81 \\
HFCs, lt7 & 0.81 \\
\end{longtable}

Figure~\ref{fig-tertcat} shows the membership following the first
generation of clusters being created for the 16 secondary categories
that exceeded this objective threshold.

\begin{figure}

{\centering \includegraphics{nts_files/figure-pdf/fig-tertcat-output-1.pdf}

}

\caption{\label{fig-tertcat}Bar chart showing the number of substances
within each tertiary category, ordered by primary and secondary category
roots. Methods to define tertiary categories are outlined in Section
2.7. Lt7, chain length less than 7; gte7, chain length greater than or
equal to 7.}

\end{figure}

Following creation of the next generation categories, there were 23
tertiary categories that met or exceeded the threshold and were
subcategorized further. The root primary categories were predominantly
from the ``Aromatic PFASs'', ``Other PFASs'', and ``PFAA precursors''
categories. Figure~\ref{fig-quartcat} reflects the quaternary categories
for the 23 that were subset further.

\begin{figure}

{\centering \includegraphics{nts_files/figure-pdf/fig-quartcat-output-1.pdf}

}

\caption{\label{fig-quartcat}Bar chart showing the number of substances
within each quaternary category, ordered by primary, secondary, and
tertiary category roots. Methods to define quaternary categories are
outlined in Section 2.7. Lt7, chain length less than 7; gte7, chain
length greater than or equal to 7.}

\end{figure}

Terminal categories were defined as either secondary or tertiary
categories that did not exceed the threshold, as well as all quaternary
categories. A total of 128 terminal categories (127 categories + 1
singleton (PFAAs, cyclic, gte7)) were ultimately derived. This
represented a trade-off in terms of the final number of terminal
categories that was a practical number to characterize the landscape of
PFAS balanced with maximizing structural similarity within the
categories themselves. The full list of 15,525 substances together with
their terminal category assignments are provided as supplementary
information. Structural similarity within categories did increase
following subcategorization, Figure~\ref{fig-ecdfs-left} shows the ECDFs
of several terminal categories which are left shifted relative to the
original ECDFs for the secondary categories
(Figure~\ref{fig-ecdfs-within}), i.e.~the pairwise distance range
decreases.

\hypertarget{sec-maxmin-all}{%
\subsection{Selection of representative
substances}\label{sec-maxmin-all}}

Whilst centroids were selected as the most representative substance from
each terminal category, there was a recognition that a single chemical
was unlikely to capture the breadth of diversity within a category.
Additional substances to capture the breadth and structural diversity
relied on the MaxMinPicker method \citep{ashton_identification_2002}.
This method was used to select up to 3 further substances in addition to
the centroid. A total of 484 substances were selected using this
approach for 121 of the terminal categories. Terminal categories with 5
or fewer members did not result in any additional substances being
selected (beyond the centroid) by the approach.
Table~\ref{tbl-maxmin-cat} lists the 7 terminal categories which had
insufficient membership to apply the MaxMinPicker approach.

\hypertarget{tbl-maxmin-cat}{}
\begin{longtable}[]{@{}ll@{}}
\caption{\label{tbl-maxmin-cat}Terminal categories for which the MaxMin
approach was not undertaken.}\tabularnewline
\toprule\noalign{}
Terminal category & Membership \\
\midrule\noalign{}
\endfirsthead
\toprule\noalign{}
Terminal category & Membership \\
\midrule\noalign{}
\endhead
\bottomrule\noalign{}
\endlastfoot
Other PFASs, cyclic, gte7, 1 & 2 \\
Other PFASs, cyclic, gte7, 2 & 4 \\
Other PFASs, cyclic, gte7, 3 & 2 \\
PFAAs, cyclic, gte7 & 1 \\
PFAAs, cyclic, lt7, 3, 1 & 4 \\
unclassified, lt7, 2.0, 1.0 & 5 \\
unclassified', lt7, 2.0, 2.0 & 2 \\
\end{longtable}

To evaluate the proportion of structural diversity captured by the
selected representative substances, the normalized cumulative minimum
distance was calculated as a function of the number of substances
selected using the MaxMinPicker method as discussed in
Section~\ref{sec-maxmin}. There were 15 terminal categories, out of the
121 terminal categories for which diverse substances were selected,
where picking 3 substances captured at least 50\% of the structural
diversity (shown in Table~\ref{tbl-structdiversity-all}).

\hypertarget{tbl-structdiversity-all}{}
\begin{longtable}[]{@{}
  >{\raggedright\arraybackslash}p{(\columnwidth - 6\tabcolsep) * \real{0.2466}}
  >{\raggedright\arraybackslash}p{(\columnwidth - 6\tabcolsep) * \real{0.2603}}
  >{\raggedright\arraybackslash}p{(\columnwidth - 6\tabcolsep) * \real{0.2466}}
  >{\raggedright\arraybackslash}p{(\columnwidth - 6\tabcolsep) * \real{0.2466}}@{}}
\caption{\label{tbl-structdiversity-all}Terminal categories for which 3
representative substance selections capture more than 50\% of the
structural diversity.}\tabularnewline
\toprule\noalign{}
\begin{minipage}[b]{\linewidth}\raggedright
Terminal category
\end{minipage} & \begin{minipage}[b]{\linewidth}\raggedright
Number of chemicals for 80\% structural diversity
\end{minipage} & \begin{minipage}[b]{\linewidth}\raggedright
Cumulative \% of Structural Diversity
\end{minipage} & \begin{minipage}[b]{\linewidth}\raggedright
Terminal Category size
\end{minipage} \\
\midrule\noalign{}
\endfirsthead
\toprule\noalign{}
\begin{minipage}[b]{\linewidth}\raggedright
Terminal category
\end{minipage} & \begin{minipage}[b]{\linewidth}\raggedright
Number of chemicals for 80\% structural diversity
\end{minipage} & \begin{minipage}[b]{\linewidth}\raggedright
Cumulative \% of Structural Diversity
\end{minipage} & \begin{minipage}[b]{\linewidth}\raggedright
Terminal Category size
\end{minipage} \\
\midrule\noalign{}
\endhead
\bottomrule\noalign{}
\endlastfoot
Other PFASs, cyclic, gte7, 4 & 3 & 84.03 & 6 \\
Polyfluoroalkyl acids, cyclic, lt7, 3 & 3 & 83.67 & 11 \\
unclassified, lt7, 1.0, 1.0 & 4 & 67.35 & 8 \\
Other PFASs, cyclic, lt7, 1.0, 2.0 & 4 & 67.14 & 9 \\
PFAAs, cyclic, lt7, 3.0, 2.0 & 5 & 64.87 & 9 \\
others, cyclic, lt7, 2.0, 3.0 & 5 & 63.47 & 10 \\
others, cyclic, lt7, 3.0, 4.0 & 5 & 62.24 & 13 \\
PFAAs, cyclic, lt7, 2.0 & 6 & 59.02 & 14 \\
others, cyclic, lt7, 2.0, 1.0 & 5 & 58.79 & 10 \\
PFAA precursors, cyclic, lt7, 1.0, 1.0 & 6 & 58.03 & 19 \\
PFAAs, cyclic, lt7, 3.0, 3.0 & 6 & 57.47 & 12 \\
Polyfluoroalkyl acids, cyclic, lt7, 4.0 & 6 & 55.93 & 12 \\
unclassified, lt7, 3.0, 2.0 & 6 & 55.75 & 11 \\
others, cyclic, lt7, 1.0, 1.0 & 6 & 55.17 & 12 \\
Polyfluoroalkyl acids, cyclic, lt7, 1.0 & 6 & 54.9 & 15 \\
\end{longtable}

Notes: Column 1 represents the number of substances that would be
required to capture 80\% of the structural diversity in the category,
Cumulative \% of Structural Diversity represents the normalized
cumulative minimum distance for up to 3 selected diverse substances.
Note the 80\% used as a threshold is purely for illustrative purposes
only.

For the largest terminal category, ``Aromatic PFASs, lt7, 2.0, 5.0'',
selecting up to 3 diverse substances only captured 0.8\% of the
structural diversity. In order to capture 80\% of the structural
diversity for this terminal category, 528 substances would need to be
selected for data collection. The number of substances that needed to be
selected from each terminal category to capture 80\% of the structural
diversity varied from 3 (as shown above in
Table~\ref{tbl-structdiversity-all}) to 528 with the median number being
30.

Figure~\ref{fig-structdiv} shows the curves of the number of diverse
selections as a function of the percentage normalized cumulative minimum
distances for 10 representative terminal categories. These vary in
steepness showing how quickly or not the structural diversity coverage
converges with number of diverse selections depending on the terminal
category of interest. Figure~\ref{fig-structdiv-hm} highlights the
difference in the number of diverse chemicals that would be needed to
capture a minimum structural diversity across each terminal category.

\begin{figure}

{\centering \includegraphics{Figure1_structural_diversity_100524.png}

}

\caption{\label{fig-structdiv}For a selection of terminal categories,
the extent to which the fraction of structural diversity is captured
relative to number of diverse chemicals selected varies.}

\end{figure}

\begin{figure}

{\centering \includegraphics{structural_diversity_hm_060824.png}

}

\caption{\label{fig-structdiv-hm}Heatmap showing the number of diverse
substances that would need to be selected to achieve a specific minimum
\% structural diversity coverage across each terminal category. A
selection of terminal categories are plotted to highlight how the number
of diverse substances needed varies. The legend color corresponds to the
number of substances needed as presented on the x axis.}

\end{figure}

Figure~\ref{fig-structdiv-corr} attempts to summarize the tradeoff of
the number of diverse chemicals (thus the centroids and MaxMin) as a
function of \% structural diversity captured across the terminal
categories.

\begin{figure}

{\centering \includegraphics{Terminal_categorisation_unconstrained_landscape_100524.png}

}

\caption{\label{fig-structdiv-corr}Lineplot showing the number of
diverse substances that would need to be selected to achieve a specific
minimum \% structural diversity coverage across all the terminal
categories. To achieve 80\% structural diversity, a total of 5929
substances would need to be selected across the various terminal
categories.}

\end{figure}

The diverse selections identified earlier for the terminal categories
reflects a pragmatism in terms of identifying a potential candidate list
of substances. As discussed later in Section~\ref{sec-tsca}, the
structural diversity captured forms one of the considerations in
selecting candidates for additional data collection relative to those
terminal categories that are data poor or contain substances that are on
the TSCA inventory.

\hypertarget{sec-physchem}{%
\subsection{Evaluation of physical state and physicochemical consistency
within terminal categories}\label{sec-physchem}}

In order to determine the nature of further data collection activities
and understanding the potential presence in different environmental
media, physical state and physicochemical information was determined for
the PFAS landscape as far as possible. For the 15,525 substances in the
PFAS landscape, 431 substances (2.8\%) could not be assigned into any
specific physical state and physicochemical designation owing to a lack
of predicted physicochemical property information. The designations of
the remaining substances are shown in Table~\ref{tbl-testtrack}.

\hypertarget{tbl-testtrack}{}
\begin{longtable}[]{@{}
  >{\centering\arraybackslash}p{(\columnwidth - 4\tabcolsep) * \real{0.4583}}
  >{\centering\arraybackslash}p{(\columnwidth - 4\tabcolsep) * \real{0.2639}}
  >{\centering\arraybackslash}p{(\columnwidth - 4\tabcolsep) * \real{0.2778}}@{}}
\caption{\label{tbl-testtrack}Number (percentage) of substances assigned
to each physical state and physicochemical designation.}\tabularnewline
\toprule\noalign{}
\begin{minipage}[b]{\linewidth}\centering
Physical state and physicochemical designation
\end{minipage} & \begin{minipage}[b]{\linewidth}\centering
Full landscape
\end{minipage} & \begin{minipage}[b]{\linewidth}\centering
TSCA active constrained landscape
\end{minipage} \\
\midrule\noalign{}
\endfirsthead
\toprule\noalign{}
\begin{minipage}[b]{\linewidth}\centering
Physical state and physicochemical designation
\end{minipage} & \begin{minipage}[b]{\linewidth}\centering
Full landscape
\end{minipage} & \begin{minipage}[b]{\linewidth}\centering
TSCA active constrained landscape
\end{minipage} \\
\midrule\noalign{}
\endhead
\bottomrule\noalign{}
\endlastfoot
A (insoluble solids) & 2060 (13.2\%) & 25 (12.6\%) \\
B (soluble solids and soluble non-volatile liquids) & 9824 (63.3\%) & 71
(35.7\%) \\
C (soluble volatile liquids/insoluble liquids and soluble gases) & 3115
(20\%) & 85 (42.7\%) \\
D (insoluble gases or highly volatile gases) & 95 (0.6\%) & 10 (5\%) \\
No designation & 431 (2.8\%) & 8 (4\%) \\
\end{longtable}

The high percentage of substances in designation C additionally raises
questions about the compatibility with most NAM-based systems. Across
the terminal categories, there was a general trend of number of
different designations increasing with size in category membership (see
Figure~\ref{fig-cat-mem}). Figure~\ref{fig-tsne-physchem} shows an
example of one of the most diverse and largest terminal categories
``Aromatic PFASs, lt7, 2.0, 5.0'' which comprises 1238 members and spans
3 of the 4 designations. Although substances predominantly lie within
designation B, there is no discernible separation between the
designations across the structural category as characterized by Morgan
fingerprints. In contrast all 96 substances belonging to terminal
category ``PolyFCA derivatives, lt7, 4.0, 3.0'' fell into designation B
(figure not shown) whereas the 58 substances in ``Aromatic PFASs, lt7,
4.0, 1.0'' fell into designations A and B. There was a positive
association between how structurally similar a terminal category was and
the consistency in physical state and physicochemical profile observed
(as reflected by the designations). However, the Morgan fingerprints
could not resolve all the differences. For the selection of potential
candidates for data collection, the physical state and physicochemical
profile remains an important consideration in concert with the
structural diversity described in Section~\ref{sec-maxmin-all}.

\begin{figure}[H]

{\centering \includegraphics{Figure9_060824.png}

}

\caption{\label{fig-tsne-physchem}UMAP projections for terminal category
a) ``Aromatic PFASs, lt7, 2.0, 5.0'' and b) ``Aromatic PFASs, lt7, 4.0,
1.0'' using Morgan chemical fingerprints with physical state and
physicochemical designations A-D overlaid.}

\end{figure}

\hypertarget{variation-of-pod-values-across-and-within-terminal-categories}{%
\subsection{Variation of POD values across and within terminal
categories}\label{variation-of-pod-values-across-and-within-terminal-categories}}

Ultimately, the terminal categories are intended to facilitate a
read-across for human health assessment. To explore the feasibility of
this further, the 25\textsuperscript{th} percentile values of oral BMDhs
were calculated using available non-cancer data for 55 substances and
repro/developmental toxicity data for 35 substances. The distributions
were plotted in a series of box plots. \emph{In vivo} toxicity data were
available for at least one chemical in 28 of the 128 terminal categories
across the two study types (28 for non-cancer, 19 for
repro/developmental). The available data allowed preliminary trends for
terminal categories to be observed where the primary root was Aromatic
PFASs, PASF-based substances, PFAAs, Polyfluoroalkanes and
Polyfluoroalkyl acids categories (see Figure~\ref{fig-pods-all} in the
supplementary information for the boxplots for both study types).

Figure~\ref{fig-pods-oral} shows boxplot and strip plots for the oral
non-cancer studies only. It appears that substances at each end of the
spectrum of chain length within a category tended to exhibit lower
toxicity, i.e., their aggregate POD is higher. The spread of POD values
within a category with greater diversity in chain length tend to span
\textasciitilde{} 1-2 orders of magnitude.

\begin{figure}

{\centering \includegraphics{ch7_POD_nrd_oral_120524.png}

}

\caption{\label{fig-pods-oral}Boxplots showing the spread of the
25\textsuperscript{th} percentile of the oral non-cancer log 10 POD
values across and within terminal categories bounded by the carbon chain
number. The box in the boxplot reflects the quartiles of the dataset,
whilst the whiskers extend to + 1.5 * inter-quartile range (IQR).
Outliers are shown as points if they exceed 1.5 * IQR.}

\end{figure}

Although the available toxicity data are limited, there does appear to
be some separation in the potency distributions between terminal
categories based on a common primary root. Inspection of
Figure~\ref{fig-pods-oral} does show a shift in potency values between
the PFAA categories with a left shift for those substances in the gte7
category vs the majority of the PFAA lt7 categories. A similar shift was
observed for the PASF-based categories. However, the relatively large
spread for some of the terminal categories suggests that additional
refinement beyond structural similarity and chain length (such as
factoring in toxicokinetic information) will likely be needed for some
terminal categories prior to broader application in a read-across
context.

\hypertarget{qualitative-mechanistic-and-toxicokinetic-designations}{%
\subsection{Qualitative mechanistic and toxicokinetic
designations}\label{qualitative-mechanistic-and-toxicokinetic-designations}}

There were six data streams with qualitative flags assigned for the
\textasciitilde150 PFAS tested as part of the research project described
in Patlewicz et al \citep{patlewicz_towards_2022} namely: 1) nuclear
receptor assays (NR); 2) developmental toxicity (zebrafish testing); 3)
DNT (developmental neurotoxicity); 4) thyroid toxicity; 5)
immunosuppression (BioMAP assays); and 6) toxicokinetics (TK).
Figure~\ref{fig-nams} profiles all the NAM flags across the different
technologies together with a stock QC flag \citep{smeltz_targeted_2023}
(Pass (red)) and a qc\_httk flag (Pass (red)).

\begin{figure}

{\centering \includegraphics{NAM_flags_091123.png}

}

\caption{\label{fig-nams}Heatmap of NAMs flags for the
\textasciitilde150 PFAS substances (\textasciitilde120 of which passed
analytical QC (qc)) tested as part of the research programme described
in Patlewicz et al \citep{patlewicz_towards_2022}. Y axis tick labels do
not capture all substances, only every 3\textsuperscript{rd} substance
by DTXSID is shown. qc = analystic QC and qc-httk analytical QC for TK;
ESR1 = Estrogen Receptor 1; PPAR = peroxisome proliferator- activated
receptor; NRF2 = nuclear factor erythroid 2-related factor 2; RXR=
retinoid X receptor; PXR=pregnane X receptor; ZF = zebrafish; DNT =
developmental neurotoxicity; DIO1, DIO2, DIO3 = Type 1,2,3 deiodinase;
IYD = iodotyrosine deiodinase; TBG = thyroxine binding globulin; TPO =
thyroid peroxidase.No data were represented as null values (white
colored), data available but no flag identified as denoted a 0 (blue
colored), 1 denoted a medium confidence flag (yellow colored) and 2 was
associated with a high confidence flag (colored in red) consistent with
the descriptions described in Table~\ref{tbl-nam}.}

\end{figure}

From Figure~\ref{fig-nams}, the first two columns represent the quality
control (QC) information. The next 5 columns represent the NR data. The
next 2 columns represent the developmental toxicity (ZF) assay and the
DNT assay. The next 8 columns represent the thyroid assay outcomes
followed by the integrated immunotoxicity flag from the BioMap assays.
The last 3 columns represent the TK flags.

A semi quantitative analysis was performed by computing which PFAS
ToxPrints were enriched for each NAM flag (excluding the TK\_Metab and
TK\_Struc\_Endo flags since this information was captured using the
predictions from the Dawson et al. \citep{dawson_2023} model). The full
set of enriched ToxPrints are provided in the supplementary information.
Those enriched PFAS ToxPrints were then used to profile the entire
landscape to provide a predicted NAM flag profile. Performance metrics
were derived to compare the actual and predicted NAM flag (see
Table~\ref{tbl-metrics}). Performance appeared weakest for the DNT and
Immune flags which had the fewest number of ToxPrints that were
enriched. It is worth noting that only \textasciitilde150 chemicals (of
which 124 substances overlapped with the PFAS landscape) were tested and
the performance of these ToxPrint signatures may change as more
substances are tested.

\hypertarget{tbl-metrics}{}
\begin{longtable}[]{@{}
  >{\raggedright\arraybackslash}p{(\columnwidth - 6\tabcolsep) * \real{0.2603}}
  >{\centering\arraybackslash}p{(\columnwidth - 6\tabcolsep) * \real{0.2466}}
  >{\centering\arraybackslash}p{(\columnwidth - 6\tabcolsep) * \real{0.2466}}
  >{\centering\arraybackslash}p{(\columnwidth - 6\tabcolsep) * \real{0.2466}}@{}}
\caption{\label{tbl-metrics}Performance metrics for enriched PFAS
ToxPrints}\tabularnewline
\toprule\noalign{}
\begin{minipage}[b]{\linewidth}\raggedright
comparison
\end{minipage} & \begin{minipage}[b]{\linewidth}\centering
ROC\_AUC\_score
\end{minipage} & \begin{minipage}[b]{\linewidth}\centering
Sensitivity
\end{minipage} & \begin{minipage}[b]{\linewidth}\centering
Specificity
\end{minipage} \\
\midrule\noalign{}
\endfirsthead
\toprule\noalign{}
\begin{minipage}[b]{\linewidth}\raggedright
comparison
\end{minipage} & \begin{minipage}[b]{\linewidth}\centering
ROC\_AUC\_score
\end{minipage} & \begin{minipage}[b]{\linewidth}\centering
Sensitivity
\end{minipage} & \begin{minipage}[b]{\linewidth}\centering
Specificity
\end{minipage} \\
\midrule\noalign{}
\endhead
\bottomrule\noalign{}
\endlastfoot
{[}NR\_ESR1, pred\_NR\_ESR1{]} & 0.67 & 0.47 & 0.87 \\
{[}NR\_PPAR, pred\_NR\_PPAR{]} & 0.78 & 0.88 & 0.68 \\
{[}NR\_NRF2, pred\_NR\_NRF2{]} & 0.82 & 1.00 & 0.64 \\
{[}NR\_PXR, pred\_NR\_PXR{]} & 0.76 & 1.00 & 0.53 \\
{[}ZF, pred\_ZF{]} & 0.66 & 0.59 & 0.74 \\
{[}DNT, pred\_DNT{]} & 0.56 & 0.14 & 0.98 \\
{[}Immune, pred\_Immune{]} & 0.55 & 0.10 & 0.99 \\
{[}TK\_PlasBind, pred\_TK\_PlasBind{]} & 0.71 & 0.78 & 0.63 \\
\end{longtable}

Predictions of half-life in humans were made for all the 15,525
substances. 69\% of the substances were predicted to have the slowest
half-life of greater than 2 months (bin 4), 14 \% in the next slowest
bin (bin 3, 1 week−2 months) and the remaining 17\% in the second
fastest bin (bin 2, 12 h−1 week).

To demonstrate the integration of the mechanistic and TK-related data
with the structural categories, two terminal categories were clustered
based on the predicted mechanistic and TK enrichment flags as shown in
Figure~\ref{fig-hmnams}. Terminal categories ``Aromatic PFASs, gte7''
and ``PFAAs, lt7, 4.0'' were first profiled against the enriched PFAS
ToxPrints in conjunction with the predicted half-lives and then
clustered to show the potential NAM profile across the entire category
and how consistent it was across the terminal category members. Terminal
category ``Aromatic PFASs, gte7'' had a far more consistent profile
whereas ``PFAAs, lt7, 4.0'' showed some differences in the half-life
predictions which might warrant further additional substances to be
identified for data collection to reflect the TK diversity.

\begin{figure}

\begin{minipage}[t]{0.50\linewidth}

{\centering 

\raisebox{-\height}{

\includegraphics{predicted_NAM_CS1.png}

}

}

\subcaption{\label{fig-cs1}Aromatic PFASs, gte7}
\end{minipage}%
%
\begin{minipage}[t]{0.50\linewidth}

{\centering 

\raisebox{-\height}{

\includegraphics{predicted_NAM_CS2.png}

}

}

\subcaption{\label{fig-cs2}PFAAs, lt7, 4.0}
\end{minipage}%

\caption{\label{fig-hmnams}Clustermaps for terminal category ``Aromatic
PFASs, gte7'' and ``PFAAs, lt7, 4.0'' to illustrate their concordance
across predicted NAM profiles.}

\end{figure}

\hypertarget{sec-tsca}{%
\subsection{Potential application to support the National PFAS Testing
Strategy (NTS)}\label{sec-tsca}}

There are several considerations that come into play when identifying
potential candidates for data collection in concert with the landscape
defined. To make the NTS actionable, one consideration was to limit the
landscape to one that was constrained by the TSCA active inventory to
increase the feasibility of being able to identify a
manufacturer/importer of the substance. A second enables the tradeoff
between the number of diverse substances to select vs capturing the
structural diversity to be more practically addressed. Herein, the scope
of terminal categories represented by the full TSCA and TSCA active
inventory and the impact this had in terms of capturing structural
diversity was evaluated. Finally, a proposal was outlined that considers
how the terminal categories could be triaged to initially focus on
terminal categories which were either data poor or contained members
that represented large exposure sources.

\hypertarget{constraining-the-landscape-to-the-tsca-active-inventory}{%
\subsubsection{Constraining the landscape to the TSCA active
inventory}\label{constraining-the-landscape-to-the-tsca-active-inventory}}

\hypertarget{tsca-inventory}{%
\paragraph{TSCA inventory}\label{tsca-inventory}}

Of the substances in the PFAS landscape, only 563 substances were
identified to be on the TSCA inventory, of which 237 were `active' and
the remaining 326 `inactive'. Active and inactive refers to the EPA's
designation of whether a substance is active in US commerce based on the
rule requiring industry to report chemicals manufactured or imported or
processed in the US over a 10 year period ending 21st June 2016. There
were 384 substances in the full landscape that matched a degradant of a
TSCA substance. Figure~\ref{fig-barplot-tsca} shows a bar chart of the
membership of the terminal categories and how that differs when
considering TSCA inventory status (overall or by active TSCA only).

\begin{figure}

{\centering \includegraphics[width=0.8\textwidth,height=\textheight]{FigureS3_100524.png}

}

\caption{\label{fig-barplot-tsca}Bar chart showing membership of
terminal categories and how that differs when constrained by TSCA
inventory or TSCA active inventory.}

\end{figure}

The largest category memberships when constrained by presence on the
TSCA inventory reflected the ``PASF-based substances, lt7'',
``PASF-based substances, lt7'' and ``PFAA precursors, gte7'' categories.
Across the terminal categories, 63\% of the categories (80 out of the
128 categories) contain members on the TSCA inventory. If only
categories containing substances that are on the TSCA active inventory
are considered, then the number of terminal categories decreases to 60,
i.e., 47\% coverage. Some of the categories where there were no examples
on the TSCA inventory were fairly large in size, examples include
several of the Aromatic PFASs categories with 174-592 members as well as
the PolyFCA derivatives and Polyfluoroalkyl acids with 186 and 151
members respectively.

\hypertarget{sec-maxmin-tsca}{%
\paragraph{Selection of representative substances in the constrained
TSCA active inventory}\label{sec-maxmin-tsca}}

Centroids were computed for the 60 terminal categories containing
substances that were on the active TSCA inventory. For 14 of these
terminal categories, membership exceeded 5, which permitted the
MaxMinPicker approach to be applied to identified further analogues. An
additional 56 analogues were selected from this constrained landscape.
Figure~\ref{fig-venn} shows the overlap in substances (centroids and
diverse) across the unconstrained and the TSCA active constrained
landscapes. The minimal overlap between the sets highlights the
limitations of using a constrained landscape, i.e., one which does not
represent the breadth of the PFAS chemistry. However, the substances on
the TSCA active inventory represent those substances that are currently
in commerce in the US and potentially represent the largest exposure
source. It is worth noting that the overlap in the venn diagram in terms
of exact substance may not reflect the overlap in structurally similar
substances.

\begin{figure}

{\centering \includegraphics{Figure7_100524.png}

}

\caption{\label{fig-venn}Venn diagram showing the overlap in substances
based on whether they were identified as additional diverse picks or
centroids in the unconstrained PFAS landscape and that constrained by
the TSCA active inventory.}

\end{figure}

An evaluation of the structural diversity captured using the centroids
and additional MaxMin substances relative to the number of substances
that would need to be selected to attain 80\% structural diversity
coverage was also undertaken in the same manner as had been performed
for the full landscape. For the 13 of the 14 categories where the MaxMin
approach had been applied, the diverse picks originally selected
captured more than 50\% of the structural diversity as shown in
Table~\ref{tbl-tsca-maxmin}. This is not so surprising given the TSCA
active set substantially limited the terminal category size and in turn
their diversity.

\hypertarget{tbl-tsca-maxmin}{}
\begin{longtable}[]{@{}
  >{\raggedright\arraybackslash}p{(\columnwidth - 6\tabcolsep) * \real{0.2432}}
  >{\raggedright\arraybackslash}p{(\columnwidth - 6\tabcolsep) * \real{0.2703}}
  >{\raggedright\arraybackslash}p{(\columnwidth - 6\tabcolsep) * \real{0.2432}}
  >{\raggedright\arraybackslash}p{(\columnwidth - 6\tabcolsep) * \real{0.2432}}@{}}
\caption{\label{tbl-tsca-maxmin}Terminal categories from the constrained
TSCA active landscape where the MaxMin approach had been applied.
Terminal categories for which 3 representative substance selections
capture more than 50\% of the structural diversity}\tabularnewline
\toprule\noalign{}
\begin{minipage}[b]{\linewidth}\raggedright
Terminal category
\end{minipage} & \begin{minipage}[b]{\linewidth}\raggedright
Number of chemicals for 80\% structural diversity
\end{minipage} & \begin{minipage}[b]{\linewidth}\raggedright
Cumulative \% of structural diversity
\end{minipage} & \begin{minipage}[b]{\linewidth}\raggedright
Terminal category size
\end{minipage} \\
\midrule\noalign{}
\endfirsthead
\toprule\noalign{}
\begin{minipage}[b]{\linewidth}\raggedright
Terminal category
\end{minipage} & \begin{minipage}[b]{\linewidth}\raggedright
Number of chemicals for 80\% structural diversity
\end{minipage} & \begin{minipage}[b]{\linewidth}\raggedright
Cumulative \% of structural diversity
\end{minipage} & \begin{minipage}[b]{\linewidth}\raggedright
Terminal category size
\end{minipage} \\
\midrule\noalign{}
\endhead
\bottomrule\noalign{}
\endlastfoot
n:2 fluorotelomer-based substances, lt7 & 4 & 79.39 & 7 \\
Other PFASs, lt7, 3.0, 1.0 & 3 & 87.82 & 6 \\
Other PFASs, cyclic, lt7, 2.0 & 4 & 75.73 & 8 \\
PASF-based substances, gte7 & 6 & 50.43 & 18 \\
PFAA precursors, gte7 & 4 & 66.79 & 19 \\
PFAA precursors, lt7, 2.0, 2.0 & 3 & 83.96 & 6 \\
PFAA precursors, lt7, 2.0, 3.0 & 2 & 97.06 & 6 \\
PFAA precursors, lt7, 3.0, 2.0 & 5 & 66.84 & 9 \\
PFAA precursors, lt7, 4.0, 2.0 & 5 & 63.06 & 11 \\
PFAAs, gte7 & 3 & 90 & 11 \\
PFAAs, lt7, 1.0 & 3 & 81.61 & 8 \\
PFAAs, lt7, 4.0 & 5 & 59.97 & 12 \\
n:2 fluorotelomer-based substances, gte7 & 1 & 100 & 6 \\
\end{longtable}

Notes: `Number of chemicals for 80\% structural diversity' represents
the number of diverse substances that would need to be selected to
capture 80\% of the structural diversity, `Cumulative \% of Structural
Diversity' reflects the structural diversity captured by the up to 3
diverse substances already made and `Terminal Category size' reflects
the size of the terminal category if constrained by the availability of
TSCA active substances.

For the largest terminal category, ``PASF-based substances, lt7'', 3
diverse substance selections captured 47.72\% of the structural
diversity. In order to capture 80\% of the structural diversity for this
terminal category, 7 substances would need to be selected for additional
data collection. The number of substances to select from each terminal
category to capture 80\% of the structural diversity varied from 1 to 7
with the median number being 4. Across the entire TSCA active space,
considering the 14 categories where a MaxMin approach could be applied
-- 55 substances would need to be selected to capture a 80\% structural
diversity. In order to capture a minimum of 80\% structural diversity
across all the TSCA active categories, at least 101 substances (the
centroids for categories where no MaxMin had been applied + MaxMin)
would be ideally selected for data collection.
Figure~\ref{fig-structdiv-tsca} summarizes the structural diversity
attained across all TSCA active terminal categories.

\begin{figure}

{\centering \includegraphics{Terminal_categorisation_constrained_landscape_110524.png}

}

\caption{\label{fig-structdiv-tsca}Lineplot of the TSCA active
constrained terminal categories as a function of number of diverse
substances selected and the minimum \% structural diversity captured
across all terminal categories.}

\end{figure}

\hypertarget{proof-of-concept-workflow-identifying-potential-candidates-for-data-collection}{%
\subsubsection{Proof of Concept Workflow: Identifying potential
candidates for data
collection}\label{proof-of-concept-workflow-identifying-potential-candidates-for-data-collection}}

The availability of toxicity data across different study types and the
presence of substances within different monitoring lists was arrayed
across the terminal categories. All oral and inhalation studies from
ToxValDB 9.5 were first retrieved. There were 76 substances with data
for one or more of the study types which were then matched on the basis
of DTXSID with substances in the PFAS landscape. The resulting table was
then transformed to produce a table where columns represented different
study types, rows were substances and cells were labelled 1 if data for
a specific study type existed for a specific substance and 0 if no data
existed. The qualitative lists reflecting various priorities,
environmental detection/discharged, and availability of analytical
methods etc. as described in Section~\ref{sec-lists} were compiled
together and transformed into a table where rows represented substances,
columns represented the different list sources and cells were populated
with a 1 or 0 to denote presence or absence on a specific list. There
were 448 substances identified across these lists but only 198 unique
substances which were then matched with substances in the PFAS
landscape. CDR status tags and Pubmed count tags were then added to the
PFAS landscape. Columns representing the toxicity study types and
various lists were grouped by terminal category to produce a new table
which reflected presence or absence of information (denoted by 1 or 0).
Study quality was not considered - only the availability of publicly
available toxicity data. The set of terminal categories were filtered to
retain only those terminal categories which contained members on the
TSCA active inventory (60 terminal categories).
Figure~\ref{fig-hm-tscaact} provides a perspective of this information,
namely the toxicity data sparsity across the categories that fall within
the scope of the TSCA active inventory as well as different
environmental monitoring efforts or discussed in the literature. The
PFAAs categories and their subcategorizations show up with data entries
which is largely unsurprising, given the extent to which PFOA and PFOS
have been studied. Note: The figure provides a landscape perspective of
the data coverage across broad structural categories which may not
entirely align with toxicological classes.

\begin{figure}[H]

{\centering \includegraphics{Figure15_130524.png}

}

\caption{\label{fig-hm-tscaact}Heatmap of toxicity data availability and
qualitative exposure and release designations. Notes: pubmed\_avail is a
tag to denote presence or absence of articles indexed in Pubmed.
PROD-Data = Production data, DISCHARGE = Discharge Monitoring data,
DRINKING\_WATER = Drinking Water (State) Data, DRINKING\_WATER-UCMR =
Drinking Water data comprising Unregulated Contaminant Monitoring Rule
data and State level monitoring data, ENV\_MEDIA = Environmental Media
data, TRI\_Waste = Toxics Release Inventory (TRI) Data Waste Managed,
TRI\_On-Site = On Site TRI Data, TRI\_Off-Site = Off Site TRI Data,
Analytical\_Mthds = PFAS with Validated Analytical Methods 533 and 537.}

\end{figure}

Each of the earlier sections in of themselves highlight different lines
of evidence that can inform the identification of potential candidates
for data collection. Note: The qualitative release designations are not
intended to be exhaustive but could be refined to factor other relevant
information data streams such as existing epidemiological studies etc.
Here, an attempt was made to demonstrate how these steps can be
integrated together to triage terminal categories and their potential
candidates for subsequent tiered data collection efforts
(Figure~\ref{fig-test-cand}). Step 1 is to consider a given terminal
category and determine whether it meets the condition of being a `data
poor category'. Data-poor in this context was to consider whether this
was a category that did not contain any members for which repeated dose
toxicity data existed (by the oral or inhalation route and with a
reported NOAEL, LOAEL, LOEL, NOEL, NEL or LEL value). Note in this
study, only repeated dose toxicity data within the publicly available
ToxVal was considered. There were 94 terminal categories out of the 128
total number of categories that met this condition.

The next step was to focus on terminal categories that overlapped with
those which contained substances that were on the TSCA inventory.

\begin{figure}

{\centering \includegraphics{Figure15_Workflow_130524.png}

}

\caption{\label{fig-test-cand}Workflow to highlight the main steps
involved in prioritizing potential candidate selection for data
collection for a given terminal category.}

\end{figure}

There were 80 terminal categories that contained substances that were on
the TSCA inventory of which 60 terminal categories contained substances
that were on the TSCA active inventory. Of the TSCA categories, 48 also
satisfied the condition of being a `data poor' category. In contrast, 31
of the TSCA active categories were `data poor'. The following step was
to consider terminal categories that contained substances that were on
different environmental monitoring (EM) lists. There were 53 terminal
categories that contained substances that were on one or more monitoring
lists or 117 if Pubmed article availability was taken into account. Of
these EM terminal categories, 21 were also overlapping with data poor
TSCA categories or 18 of the data poor TSCA active categories. The 18
terminal categories included ``Aromatic PFASs, gte7'', ``Other PFASs,
lt7, 2.0, 2.0'', ``PFAA precursors, lt7, 1.0, 2.0'', ``PolyFCA
derivatives, gte7'', ``Polyfluoroalkanes, gte7'' and ``unclassified,
lt7, 2.0, 1.0''. Note: Only 2 of these categories met the condition to
apply the MaxMin approach - taking into account those categories for
which 3 substances would need to be selected to achieve 80\% structural
diversity, the remaining 16 categories would be limited to selecting the
centroid.

For a category that satisfied all these conditions, the next step would
be identify the representative substances characterizing the category
(namely the centroid and MaxMin substances and check whether any were on
the TSCA inventory). If none of these were on the inventory, then the
next step would be to check whether the next closest match to the
centroid was on the inventory. If not, the next steps would be to
identify the centroid and MaxMin substances from either the TSCA
constrained landscape or the TSCA active constrained landscape for that
terminal category. Figure~\ref{fig-test-cand} summarizes these steps in
a conceptual workflow.

For illustrative purposes, terminal category ``PFAA precursors, lt7,
2.0, 3.0'' was identified that met the conditions of being a data poor
category, containing members on the TSCA active inventory and containing
members on various environmental monitoring lists, discharge and TRI
lists. This terminal category comprises 56 members. If the category were
constrained by TSCA active substances only, the category size would be
reduced to 6 members of which 2 substances would capture 80\% of its
structural diversity. The centroid, DTXSID70884511 was on the TSCA
inventory but the TSCA active centroid DTXSID60880406 could be selected.
Figure~\ref{fig-case-study} shows an UMAP projection
\citep{mcinnes2020umap} with the centroid, MaxMin and TSCA centroid
substances shown for illustrative purposes to highlight their relative
positions in the structural space captured within the terminal category.

\begin{figure}[H]

{\centering \includegraphics{Figure15_case_study_130524.png}

}

\caption{\label{fig-case-study}UMAP projection of terminal category
``PFAA precursors, lt7, 2.0, 3.0'' with its (TSCA active) centroid and
MaxMin substances shown.}

\end{figure}

\hypertarget{operationalizing-the-terminal-categories-for-re-use-1}{%
\subsection{Operationalizing the terminal categories for
re-use}\label{operationalizing-the-terminal-categories-for-re-use-1}}

There were 11 terminal categories that had memberships of less than 10
which were aggregated into one miscellaneous group. A random forest
classifier was trained to assignment membership of a substance into one
of 118 categories (117 terminal categories + a miscellaneous category).
The features used were Morgan fingerprints, chain length and primary
category assigments whereas the labels were the terminal category names.
A random forest classifier with default settings applied as part of a
5-fold stratified CV procedure gave rise to a mean balanced accuracy of
0.808 (std 0.014). A randomized search CV procedure using a range of
different hyperparameters found the highest balanced accuracy (BA) to be
with 400 trees, a minimum split size of 2, a minimum number of data
points allowed in a leaf node to be 2, a maximum features to be the
square root of the number of features, and a balanced subsample class
weight. The CV mean balanced accuracy was determined to be 0.845 (std
0.013). This model was then applied to the test set that had been held
out. The balanced accuracy of the test set was determined to be 0.857.
The BA varied across the terminal categories with a median value of
0.973 and a minimum value of 0.50. The worse performing categories were
the unclassified categories namely ``unclassified, lt7, 3.0, 1.0'',
``unclassified, lt7, 3.0, 2.0'', ``unclassified, lt7, 3.0, 3.0'' and
``unclassified, lt7, 1.0, 3.0'' all of which had BA less than or equal
to 0.5. There were 8 categories with a BA less than 0.75 (of these 4 had
a BA less than or equal to 0.5). The top 5 highest performing categories
were ``Aromatic PFASs, lt7, 2.0, 3.0'', ``others, cyclic, lt7, 3.0,
3.0'', ``others, cyclic, lt7, 1.0, 1.0'', ``PFAA precursors, cyclic,
lt7, 1.0, 1.0'' and ``PFAAs, cyclic, lt7, 2.0''. The full test set
performance scores and the feature importances are provided in the
supplementary information. Figure~\ref{fig-rfc} shows the BA and recall
scores across the terminal categories.

\hypertarget{conclusions}{%
\section{Conclusions}\label{conclusions}}

EPA was directed by Congress to develop a process for prioritizing which
PFAS or classes of PFAS should be subject to additional research efforts
based on potential for human exposure, toxicity, and other available
information. Herein, we describe an approach that can be used to create
a relevant PFAS landscape using the TSCA section 8(a)(7) rule definition
to continue the efforts initiated in the National PFAS Testing Strategy.

A landscape of 13,054 PFAS substances was first created based on the
structural definitions outlined in the TSCA rule. The landscape was then
augmented with 2484 simulated degradation products of PFAS substances on
the TSCA inventory using the Catalogic expert system. Adding simulated
degradates was intended to enrich the landscape by substances that might
be expected to be found in the environment from existing substances in
commerce. The simulated degradation products were derived from an expert
system which includes training set substances that are PFAS; however a
full characterization of the model relative to the PFAS landscape was
not feasible as some of the training set was proprietary in nature. For
the portion of training set substances that could be evaluated -- there
was a minimal overlap in datasets as shown in Figure~\ref{fig-miti}. The
robustness of the simulated degradation products is a limitation in the
approach and requires additional work but a pragmatic one given the
absence of data to refine and improve the model further.

The 15,525 substances in the PFAS landscape were then grouped into 128
terminal structural categories based on a stepwise process that combined
OECD functional categories, chain length and structural similarity. The
use of a chain length threshold of 7 was a pragmatic choice to help
identify persistent long chain substances, though the subcategorization
using this threshold may be best suited for straight chain linear PFAS.
Some of the terminal categories were very large and structurally diverse
whilst others showed much greater structural similarity highlighting
both the complexity of the PFAS structural landscape and local areas of
homogeneity.

A method was developed to select the most representative substance
(centroid) and other substances (MaxMin) to capture the structural
diversity of each terminal category and guide data collection efforts. A
substantial number of representative and diverse substances
(\textasciitilde6000) would be required to capture 80\% percent of
structural diversity in the terminal categories for the unconstrained
landscape. Significantly fewer representative and diverse substances
(101) would be required to capture 80\% percent of structural diversity
in the terminal categories for the TSCA constrained landscape (though if
ToxVal data availability was factored in, this would reduce to 76
substances). The difference in utilizing the unconstrained and TSCA
constrained landscape highlights the challenges in data collection to
address future and theoretical data gaps versus those data gaps that
exist amongst substances known to be in commerce.

Publicly available \emph{in vivo} data across the terminal categories
were used to evaluate whether read-across could be potentially viable
based on the variation of the \emph{in vivo} data itself. A
25\textsuperscript{th} percentile of the derived human equivalent BMDs
served as surrogate POD value for a given substance. Not all terminal
categories were associated with toxicity data but for those categories,
the following insights were noted; substances with very short or long
carbon chain length within a category tended to exhibit lower toxicities
(i.e., higher PODs), but the spread of PODs within a category could be
large particularly for diverse categories based on carbon chain length,
spanning 1-2 orders of magnitude or more. In addition to the shift in
potency between terminal categories containing longer vs shorter chain
lengths, there was also a shift between terminal categories with
different functional groups e.g.~Aromatic PFASs tended to be less potent
vs.~PFAAs. The variability of \emph{in vivo} PODs from traditional
toxicity tests within some of the terminal categories suggests that
structural considerations may not be sufficient for performing
read-across without additional data collection.

Information from NAMs were layered on the terminal categories to help
identify potentially distinct mechanistic and TK-related subgroups. The
NAM information was intended to help refine terminal categories, guide
candidate selection, and inform data collection efforts. Currently, the
terminal categories and candidates for data collection are primarily
identified based on chemical structural considerations; however, other
factors such as toxicokinetics, hazard, and modes-of-action are also
important when considering a category and read-across approach
\citep{patlewicz_towards_2023}. To incorporate these other factors, the
current process for selecting representative and diverse substances
(i.e., centroid, MaxMin) could also be applied to the mechanistic and
TK-related subgroups to select candidates for data collection as well as
inform the types of tests that may be useful in characterizing the
hazards associated with a specific terminal category. For example,
subgroups predicted to have endocrine-related activity may benefit from
developmental and reproductive tests, whilst those predicted to have
significant cross-species TK differences may have limited benefits from
rodent-based TK studies. While the current NAM data and enrichment flags
are limited, the performance of these models will improve over time as
additional substances are tested. Similar approaches have been proposed
or incorporated into case studies, but not operationalized for large
groups of chemicals
\citep{escher_towards_2019, webster_2019, oecd_iata_2018}. Information
from environmental measurements and release, traditional toxicity data,
and chemical properties (physical state and physicochemical properties)
were also layered on the terminal categories to help identify priorities
for further data collection efforts.

The landscape of PFAS substances is substantially large and diverse with
limited human health data. A category approach enables strategic data
collection with the longer-term goal of enabling read-across within a
particular category. In an effort to address this goal, a stepwise
systematic process was developed to group the substances using a
combination of OECD primary categories, sequential fluorinated carbon
chain length, and structural similarity. The process attempted to
balance maximizing structural similarity within each category relative
to a manageable number of categories. Within each category,
representative substances to capture the structural diversity were
identified to guide data collection efforts. A substantial number of
substances were required to capture a large percentage of structural
diversity in the terminal categories for the full PFAS landscape, whilst
a significantly fewer number were needed to capture the structural
diversity for the TSCA active constrained landscape. The difference in
utilizing the full and TSCA active constrained PFAS landscape highlights
the challenges in data collection to address future and theoretical data
gaps versus those data gaps that exist amongst substances known to be in
commerce. To assist in prioritizing the categories for data collection,
information from environmental measurements and release, traditional
toxicity data, and exposure considerations based on chemical properties
were incorporated to focus on those categories of greatest need. The
variability in POD values from existing traditional \emph{in vivo}
toxicity tests within some of the terminal categories suggests that the
structural considerations may not be sufficient for performing
read-across without additional data collection. TK information is
another important factor that can help resolve the variability observed.
It should also be noted that POD values are one of many considerations
-- a reference dose effect will likely vary across similar chemicals
within a category even if they exhibit similar hazard profiles.
Information from NAMs were used to help refine the terminal categories
based on potentially distinct mechanistic and TK-related subgroups and
inform the types of data collection activities that may be required.
Finally, a machine learning modelling approach was applied in an attempt
to build a predictive model to operationalize the terminal categories
developed, such that new PFAS not already captured in the landscape
could be profiled and assigned to their most probable terminal category.
The methods developed for categorizing the PFAS landscape, selecting
representative substances, refining categories based on mechanistic and
TK information, and prioritizing categories for data collection provide
a robust foundation to aid EPA in addressing the significant challenges
associated with evaluating the environmental and human health impacts of
this class of chemicals. The methods and associated categories are
flexible in accommodating additional data as it is generated and may
evolve as the scientific knowledge grows.

\hypertarget{acknowledgements}{%
\section{Acknowledgements}\label{acknowledgements}}

The authors thank Drs Meghan Tierney, Louis (Gino) Scarano, Charles Lowe
and Nathaniel Charest for their thoughtful reviews of this manuscript.
We also thank the anonymous reviewers for their comments.

\hypertarget{funding}{%
\subsection*{Funding}\label{funding}}
\addcontentsline{toc}{subsection}{Funding}

The work presented in this manuscript was solely supported by
appropriated funds of the US Environmental Protection Agency (US EPA).

\hypertarget{disclaimer}{%
\subsection*{Disclaimer}\label{disclaimer}}
\addcontentsline{toc}{subsection}{Disclaimer}

The approach presented and the scientific views discussed in this
manuscript are those of the authors and do not necessarily reflect final
policies of the US Environmental Protection Agency (US EPA). Any mention
of trade names, manufacturers or products does not imply an endorsement
by the U.S. Government or the EPA. The EPA and its employees do not
endorse any commercial products, services, or enterprises.

\hypertarget{references}{%
\section*{References}\label{references}}
\addcontentsline{toc}{section}{References}

\renewcommand{\bibsection}{}
\bibliography{PFAS.bib}

\newpage{}

\newpage
\appendix
\renewcommand{\thefigure}{A\arabic{figure}}
\renewcommand{\thetable}{A\arabic{table}}
\setcounter{figure}{0}
\setcounter{table}{0}

\hypertarget{supplementary-information}{%
\section{Supplementary information}\label{supplementary-information}}

\hypertarget{sec-supp}{%
\subsection{Evaluating the feasibility of subdividing the primary
categories}\label{sec-supp}}

Corina Symphony on the command line (licensed from Molecular Networks
GmBH and Altamira LLC) was used to compute the 129 PFAS ToxPrints
\citep{richard_new_2023}. The Fisher's exact test was used to compute an
odds ratio and associated p value for each PFAS ToxPrint relative to the
OECD primary category designation. This was comparable with the
methodology discussed in Wang et al. \citep{wang_high-throughput_2019}.
A PFAS ToxPrint was considered enriched if it had an odds ratio greater
than or equal to 3, a one-sided Fishers exact p-value less than 0.05
(probability value of the odds ratio being greater than 1) and the
number of True Positives (TP) was determined to be greater than or equal
to 3. For the ``unclassified'' primary category, the top enriched
ToxPrints were PFAS bond and chain features including
pfas\_bond:S(=O)O\_sulfonicAcid\_acyclic\_(chain)\_SCF,
pfas\_chain:FT\_n1\_OP, pfas\_chain:FT\_n2\_OP the latter represent
fluorotelomer chains with either 1 or 2 CH2 units and an
organophosphorus terminus. On the otherhand, the ``PFAA precursors'' had
alcohols and carbonyls as enriched functional groups
(pfas\_bond:COH\_alcohol\_pri-alkyl\_CF,
pfas\_bond:CC(=O)C\_ketone\_generic\_CF). The intention was to explore
whether certain types of features were specifically enriched in these
broad primary categories to consider subcategorizing them to reduce the
starting membership. The full set of enrichments for all primary
categories are provided as a separate data file. \newpage{}

\hypertarget{supplementary-figures}{%
\subsection*{Supplementary Figures}\label{supplementary-figures}}
\addcontentsline{toc}{subsection}{Supplementary Figures}

\begin{figure}

{\centering \includegraphics{FigureS9_210524.png}

}

\caption{\label{fig-miti}Overlap of MITI training data substances with
TSCA substances using Morgan chemical fingers and represented in a UMAP
plot}

\end{figure}

\begin{figure}

{\centering \includegraphics{FigureS1_100524.png}

}

\caption{\label{fig-ecdfs-within}ECDFs of the within categories based on
the chain length threshold of 7}

\end{figure}

\newpage{}

\begin{figure}

{\centering \includegraphics{FigureS2_100524.png}

}

\caption{\label{fig-ecdfs-bet}EDCFs for selected between category
combinations for carbon chain length categories}

\end{figure}

\begin{figure}

{\centering \includegraphics{FigureS4_shiftECDF_100524.png}

}

\caption{\label{fig-ecdfs-left}EDCFs for selected terminal categories to
demonstrate left shift in pairwise distance}

\end{figure}

\begin{figure}

{\centering \includegraphics{FigureS5_120524.png}

}

\caption{\label{fig-cat-mem}Correlation between terminal categories with
large membership size and the number of designations represented amongst
their memberships}

\end{figure}

\begin{figure}

{\centering \includegraphics{Figure11_oral_PODs_120524.png}

}

\caption{\label{fig-pods-all}Boxplots of the variation of
25\textsuperscript{th} percentiles of point of departure values (PODs)
from non-cancer and repro/developmental studies. The box in the boxplot
reflects the quartiles of the dataset, whilst the whiskers extend to +
1.5 * inter-quartile range (IQR). Outliers are shown as points if they
exceed 1.5 * IQR. The repro/developmental boxplot is shown \ul{below}
the non-cancer boxplot for a given terminal category.}

\end{figure}

\begin{figure}

{\centering \includegraphics{NAM_flags_cats_130524.png}

}

\caption{\label{fig-nams-hm}Heatmap of NAM flags for substances tested
that overlap with the PFAS inventory}

\end{figure}

\begin{figure}

{\centering \includegraphics[width=0.6\textwidth,height=\textheight]{classification_rfc_150524.png}

}

\caption{\label{fig-rfc}Barplot of the performance scores for the random
forest classification model as applied to the hold out set of terminal
categories}

\end{figure}




\end{document}
